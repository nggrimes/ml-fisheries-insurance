% Options for packages loaded elsewhere
\PassOptionsToPackage{unicode}{hyperref}
\PassOptionsToPackage{hyphens}{url}
\PassOptionsToPackage{dvipsnames,svgnames,x11names}{xcolor}
%
\documentclass[
  letterpaper,
  DIV=11,
  numbers=noendperiod]{scrartcl}

\usepackage{amsmath,amssymb}
\usepackage{iftex}
\ifPDFTeX
  \usepackage[T1]{fontenc}
  \usepackage[utf8]{inputenc}
  \usepackage{textcomp} % provide euro and other symbols
\else % if luatex or xetex
  \usepackage{unicode-math}
  \defaultfontfeatures{Scale=MatchLowercase}
  \defaultfontfeatures[\rmfamily]{Ligatures=TeX,Scale=1}
\fi
\usepackage{lmodern}
\ifPDFTeX\else  
    % xetex/luatex font selection
\fi
% Use upquote if available, for straight quotes in verbatim environments
\IfFileExists{upquote.sty}{\usepackage{upquote}}{}
\IfFileExists{microtype.sty}{% use microtype if available
  \usepackage[]{microtype}
  \UseMicrotypeSet[protrusion]{basicmath} % disable protrusion for tt fonts
}{}
\makeatletter
\@ifundefined{KOMAClassName}{% if non-KOMA class
  \IfFileExists{parskip.sty}{%
    \usepackage{parskip}
  }{% else
    \setlength{\parindent}{0pt}
    \setlength{\parskip}{6pt plus 2pt minus 1pt}}
}{% if KOMA class
  \KOMAoptions{parskip=half}}
\makeatother
\usepackage{xcolor}
\setlength{\emergencystretch}{3em} % prevent overfull lines
\setcounter{secnumdepth}{5}
% Make \paragraph and \subparagraph free-standing
\makeatletter
\ifx\paragraph\undefined\else
  \let\oldparagraph\paragraph
  \renewcommand{\paragraph}{
    \@ifstar
      \xxxParagraphStar
      \xxxParagraphNoStar
  }
  \newcommand{\xxxParagraphStar}[1]{\oldparagraph*{#1}\mbox{}}
  \newcommand{\xxxParagraphNoStar}[1]{\oldparagraph{#1}\mbox{}}
\fi
\ifx\subparagraph\undefined\else
  \let\oldsubparagraph\subparagraph
  \renewcommand{\subparagraph}{
    \@ifstar
      \xxxSubParagraphStar
      \xxxSubParagraphNoStar
  }
  \newcommand{\xxxSubParagraphStar}[1]{\oldsubparagraph*{#1}\mbox{}}
  \newcommand{\xxxSubParagraphNoStar}[1]{\oldsubparagraph{#1}\mbox{}}
\fi
\makeatother


\providecommand{\tightlist}{%
  \setlength{\itemsep}{0pt}\setlength{\parskip}{0pt}}\usepackage{longtable,booktabs,array}
\usepackage{calc} % for calculating minipage widths
% Correct order of tables after \paragraph or \subparagraph
\usepackage{etoolbox}
\makeatletter
\patchcmd\longtable{\par}{\if@noskipsec\mbox{}\fi\par}{}{}
\makeatother
% Allow footnotes in longtable head/foot
\IfFileExists{footnotehyper.sty}{\usepackage{footnotehyper}}{\usepackage{footnote}}
\makesavenoteenv{longtable}
\usepackage{graphicx}
\makeatletter
\newsavebox\pandoc@box
\newcommand*\pandocbounded[1]{% scales image to fit in text height/width
  \sbox\pandoc@box{#1}%
  \Gscale@div\@tempa{\textheight}{\dimexpr\ht\pandoc@box+\dp\pandoc@box\relax}%
  \Gscale@div\@tempb{\linewidth}{\wd\pandoc@box}%
  \ifdim\@tempb\p@<\@tempa\p@\let\@tempa\@tempb\fi% select the smaller of both
  \ifdim\@tempa\p@<\p@\scalebox{\@tempa}{\usebox\pandoc@box}%
  \else\usebox{\pandoc@box}%
  \fi%
}
% Set default figure placement to htbp
\def\fps@figure{htbp}
\makeatother
% definitions for citeproc citations
\NewDocumentCommand\citeproctext{}{}
\NewDocumentCommand\citeproc{mm}{%
  \begingroup\def\citeproctext{#2}\cite{#1}\endgroup}
\makeatletter
 % allow citations to break across lines
 \let\@cite@ofmt\@firstofone
 % avoid brackets around text for \cite:
 \def\@biblabel#1{}
 \def\@cite#1#2{{#1\if@tempswa , #2\fi}}
\makeatother
\newlength{\cslhangindent}
\setlength{\cslhangindent}{1.5em}
\newlength{\csllabelwidth}
\setlength{\csllabelwidth}{3em}
\newenvironment{CSLReferences}[2] % #1 hanging-indent, #2 entry-spacing
 {\begin{list}{}{%
  \setlength{\itemindent}{0pt}
  \setlength{\leftmargin}{0pt}
  \setlength{\parsep}{0pt}
  % turn on hanging indent if param 1 is 1
  \ifodd #1
   \setlength{\leftmargin}{\cslhangindent}
   \setlength{\itemindent}{-1\cslhangindent}
  \fi
  % set entry spacing
  \setlength{\itemsep}{#2\baselineskip}}}
 {\end{list}}
\usepackage{calc}
\newcommand{\CSLBlock}[1]{\hfill\break\parbox[t]{\linewidth}{\strut\ignorespaces#1\strut}}
\newcommand{\CSLLeftMargin}[1]{\parbox[t]{\csllabelwidth}{\strut#1\strut}}
\newcommand{\CSLRightInline}[1]{\parbox[t]{\linewidth - \csllabelwidth}{\strut#1\strut}}
\newcommand{\CSLIndent}[1]{\hspace{\cslhangindent}#1}

\usepackage{booktabs}
\usepackage{longtable}
\usepackage{array}
\usepackage{multirow}
\usepackage{wrapfig}
\usepackage{float}
\usepackage{colortbl}
\usepackage{pdflscape}
\usepackage{tabu}
\usepackage{threeparttable}
\usepackage{threeparttablex}
\usepackage[normalem]{ulem}
\usepackage{makecell}
\usepackage{xcolor}
\addtokomafont{disposition}{\rmfamily}
\KOMAoption{captions}{tableheading}
\makeatletter
\@ifpackageloaded{caption}{}{\usepackage{caption}}
\AtBeginDocument{%
\ifdefined\contentsname
  \renewcommand*\contentsname{Table of contents}
\else
  \newcommand\contentsname{Table of contents}
\fi
\ifdefined\listfigurename
  \renewcommand*\listfigurename{List of Figures}
\else
  \newcommand\listfigurename{List of Figures}
\fi
\ifdefined\listtablename
  \renewcommand*\listtablename{List of Tables}
\else
  \newcommand\listtablename{List of Tables}
\fi
\ifdefined\figurename
  \renewcommand*\figurename{Figure}
\else
  \newcommand\figurename{Figure}
\fi
\ifdefined\tablename
  \renewcommand*\tablename{Table}
\else
  \newcommand\tablename{Table}
\fi
}
\@ifpackageloaded{float}{}{\usepackage{float}}
\floatstyle{ruled}
\@ifundefined{c@chapter}{\newfloat{codelisting}{h}{lop}}{\newfloat{codelisting}{h}{lop}[chapter]}
\floatname{codelisting}{Listing}
\newcommand*\listoflistings{\listof{codelisting}{List of Listings}}
\makeatother
\makeatletter
\makeatother
\makeatletter
\@ifpackageloaded{caption}{}{\usepackage{caption}}
\@ifpackageloaded{subcaption}{}{\usepackage{subcaption}}
\makeatother

\usepackage{bookmark}

\IfFileExists{xurl.sty}{\usepackage{xurl}}{} % add URL line breaks if available
\urlstyle{same} % disable monospaced font for URLs
\hypersetup{
  pdftitle={Finding suitable weather indices for novel fisheries index insurance},
  pdfauthor={Nathaniel Grimes},
  pdfkeywords={Index Insurance, Fisheries, Machine Learning},
  colorlinks=true,
  linkcolor={blue},
  filecolor={Maroon},
  citecolor={Blue},
  urlcolor={Blue},
  pdfcreator={LaTeX via pandoc}}


\title{Finding suitable weather indices for novel fisheries index
insurance}
\usepackage{etoolbox}
\makeatletter
\providecommand{\subtitle}[1]{% add subtitle to \maketitle
  \apptocmd{\@title}{\par {\large #1 \par}}{}{}
}
\makeatother
\subtitle{Working Paper not for Distribution}
\author{Nathaniel Grimes}
\date{2025-10-06}

\begin{document}
\maketitle
\begin{abstract}
Index insurance is a financial tool gaining traction for application in
fisheries. However, identifying weather indicies to serve as triggers
remains a significant barrier to widespread implementation. Weak
correlations between weather variables and catch will lead to
substaintial basis risk. This paper is the first to empirically evaluate
the feasibility of index insurance for fisheries. I use the Market Squid
Fishery in California as a case study to test insurance contracts built
on five different prediction models with twelve weather variables. Only
three weather variables in single index contracts improved fisher
welfare at an actuarially fair premium. Only the index with krill
abundance was profitable for insurance companies. All four multivariatve
models improved utility from a range of 5\%-35\% at actuarially fair
premiums. When charging a competitive market premium, the random forest,
regularized random forest, and support vector machine were profitable
for insurance. Index insurance is feasible in a private market, and
provides fishers with protection against environmental variability.
\end{abstract}

\renewcommand*\contentsname{Table of contents}
{
\hypersetup{linkcolor=}
\setcounter{tocdepth}{3}
\tableofcontents
}

\section{Introduction}\label{introduction}

Fishers face enormous environmental variability that impacts their
livelihoods. Extreme weather events that degrade stock health such as
stagnant upwelling or marine heatwaves can lead to significant losses in
fishery productivity (Smith \emph{et al.} 2023; Szuwalski \emph{et al.}
2023; Villaseñor-Derbez \emph{et al.} 2024). These shocks can devastate
fishing communities, leading to income loss and food insecurity (Holland
and Leonard 2020; Jardine \emph{et al.} 2020). The need for financial
tools to protect fishers from these shocks has never been greater
(Mumford \emph{et al.} 2009; Sethi 2010; Kasperski and Holland 2013;
Sumaila \emph{et al.} 2021). Index insurance has risen as a leading
candidate to protect fishing communities during such disasters, but
accurately identifying suitable weather indices for fisheries has
impeded widespread deployment (Watson \emph{et al.} 2023; Hobday
\emph{et al.} 2025). This paper is the first to empirically construct
index insurance contracts to protect against fluctuations in fishery
biological productivity.

Index insurance is a financial product that pays out when an
independently verified index, such as sea surface temperature, falls
below a predetermined threshold called the trigger. The index is ideally
chosen to be highly correlated with the asset being insured. In
fisheries, catch or yield will be the primary underwritten asset. The
trigger represents a critical value where the underlying asset is
expected to suffer a loss. Fishers pay a premium that transfers income
from good periods to bad periods. Index insurance is preferable to
indemnity because it is faster to administer, cheaper to monitor, and
avoids moral hazards (Barnett and Mahul 2007).

The COAST program is the only extant index insurance product for
fisheries (Sainsbury \emph{et al.} 2019). It is a pilot program that
insures small island nations in the Caribbean from hurricanes. It uses
indices of wave height, wind speed, and storm surge to initiate payouts
once they fall below the trigger. There is growing interest to expand
index insurance to cover other environmental shocks, but significant
market barriers have limited widespread implementation (Watson \emph{et
al.} 2023; Hobday \emph{et al.} 2025).

The primary market barrier is empirically constructing contracts that
sufficiently reduce basis risk. Basis risk is the potential for a
policyholder's actual loss to not align with the contract payouts. For
example, if a fisher suffers poor catch, but the weather index does not
exceed the trigger, then the fisher will not receive payouts. The
opposite may occur where insurance companies payout despite fishers
receiving high catch. Basis risk lowers demand for index insurance and
leads to exorbitant premium rates (Binswanger-Mkhize 2012; Clarke 2016;
Clement \emph{et al.} 2018).

Identifying triggers and indices that are highly correlated to loss is
the best way to eliminate basis risk (Jensen \emph{et al.} 2019).
Agricultural index insurance has refined techniques to build and price
contracts over the last 20 years (Jensen \emph{et al.} 2016). The most
common design strategy is to match individual farms to the nearest
meteorological stations, then use linear regression to model
yield-weather distributions (Miranda and Farrin 2012; Spicka and Hnilica
2013; Conradt \emph{et al.} 2015; Dalhaus \emph{et al.} 2018; Benso
\emph{et al.} 2023). Premiums are then calculated based on the payout
distribution of prior years for a given contract.

However, fisheries possess more difficult data and empirical challenges
than agriculture. Predicting fishery output from weather variables is
notoriously difficult. It is widely established that climate and weather
affect fish populations (Lehodey \emph{et al.} 2006), but even the most
advance fishery stock assessment models use little to no environmental
data for projections (Privitera-Johnson and Punt 2020). Spatial
distributions of fisheries can cover large areas making it unclear where
and how much specific weather variables impact biological abundance.
Additionally, if weather to catch connections can be established, they
are often nonlinear and interact with other variables. The biological
dynamics of fish reproduction can lead to lower stock health persisting
for years after a negative shock (Hilborn \emph{et al.} 2003).

Management is also unique challenge to navigate empirically that is
absent from agriculture insurance. Good fisheries management provides
the necessary data to model weather to catch dependency, but can also
sever the connection. For example, many fisheries in the United States
transitioned from open access to individual transferable quota systems
in the early 2000s (Costello and Polasky 2008). Distortions in catch
would be attributable to shifts in management regimes rather than
weather. In ensuing years after the implementation of quota systems,
catch would then become contingent on management choices instead of
weather.

Despite these challenges, innovations in fishery and oceanographic data
collection, and increased demand by fishers for assistance to combat
against increasing environmental variability, suggest a new opportunity
has arisen to reconsider the use of index insurance in fisheries {[}TNC
Study{]}. New methods including advance machine learning algorithms
could better capture the complex, nonlinear relationships between
weather and fishery productivity.

This paper is the first to empirically evaluate the feasibility of index
insurance for fisheries using established univariate, linear methods and
novel machine learning techniques. I use the Market Squid Fishery in
California as a case study to test insurance contracts built on five
different prediction models with twelve weather variables. I evaluate
whether the contracts improve fisher welfare by determining their
marginal willingness to pay. Through the marginal willingness to pay, I
assess whether insurance companies could feasibly offer contracts within
a private market to remain solvent.

Only one index, krill abundance, in univariate linear models improved
fisher welfare at the market premium. All four multivariate models
improved utility from a range of 5\%-35\% improvement at actuarially
fair rates. When charging a competitive market premium, the random
forest, regularized random forest, and support vector machine contracts
were profitable for insurance companies. The leading variable of
influence in the nonlinear models included upwelling measures, regional
El Nino metrics, and krill abundance. Communicating to fishers the
relative influence of each variable is crucial to build trust and
transparency when using complex models. Overall, there is potential for
index insurance to be feasible in a private market and provide fishers
with protection against environmental variability.

These results contribute to three literatures. Fisheries risk management
includes formal and informal means of mitigating risk. However,
developing formal financial tools to manage fisheries risk as been
limited. Prior studies have explored the theoretical application of
insurance in fisheries (Herrmann \emph{et al.} 2004; Mumford \emph{et
al.} 2009; Sethi 2010; Watson \emph{et al.} 2023; Hobday \emph{et al.}
2025), but few have empirically tested the feasibility of such programs.
Hobday \emph{et al.} (2025) provides a compelling case for index
insurance in fisheries as well as a comprehensive review of the
potential barriers. My study provides an evaluation of a new formal risk
mitigation tools that could be used by fishers to manage environmental
risk.

Within the index insurance literature, there has been a growing
application of machine learning to improve index design (Feng \emph{et
al.} 2019; Cesarini \emph{et al.} 2021). Payout schedules limited to
linear payouts introduce unnecessary basis risk that can be mitigated
with neural networks directly predicting the triggers (Chen \emph{et
al.} 2024). Machine learning models can also improve estimation of the
crops yields (Klompenburg \emph{et al.} 2020). Improved estimation of
crop yields leads to better insurance performance (Schmidt \emph{et al.}
2022). My study further demonstrates the value of machine learning for
index design, but reinforces the challenge of communicating complex
models to policyholders and additional considerations that influence
pricing.

The application of machine learning is growing in fisheries as
researchers explore data questions beyond formal stock assessments.
Neural networks are able to predict fishing activity in the global fleet
using vessel GPS data (Souza \emph{et al.} 2016). Ensemble models built
through combinations of random forests, boosted trees, and dynamic
linear models improve Bristol Bay sockeye salmon forecasts by 15\%
compared to a standard lagged regression model (Ovando \emph{et al.}
2022). Environmental variables of importance to groundfish populations
in Alaska were uncovered using single index varying coefficient models
regularized with LASSO (Correia 2021). Random Forests models better
predict fish catch in Indonesia than traditional linear models (Rahman
\emph{et al.} 2022). Random Forests models have also been used to
predict fishery catch in the California Market Squid Fishery to isolate
important biological and ecological variables (Akselrud 2024). My study
builds on this literature by demonstrating the value of machine learning
for fisheries even in small data sets.

The rest of the paper is structured as follows. Section~\ref{sec-ins}
describes the design of index insurance contracts used in this study as
well as showing how imperfect prediction models lead to basis risk.
Section~\ref{sec-methods} describes the algorithms used to predict
fishery productivity, and how variables of importance were identified.
Section~\ref{sec-fish} descrbies the charactericis of the market squid
fishery that make it suitable for index insurance.
Section~\ref{sec-data} includes the data collection, transformations,
and sources. Fisheries data comes from newly open-access sources
provided by the California Department of Fish and Wildlife. Weather
variables are collected from a variety of sources to explore a wide
range of potential indices. Section~\ref{sec-results} presents the
primary findings of the paper including the performance of all models.
Future steps and considerations for broader implementation of fishery
index insurance are outlined in Section~\ref{sec-discussion}.

\section{Insurance Framework}\label{sec-ins}

Index insurance is designed to protect policyholders from suffering
losses below critical yield thresholds. In fisheries index insurance,
actuaries must first estimate how fishery harvest is affected by weather
variables. For example, fisheries harvest, \(y\) can be modeled with a
simple technology function where fisher inputs, \(x\) interact with a
stock of biomass, \(B\), which is a function of random weather shocks,
\(\tilde{w}\) as shown in Equation~\ref{eq-fishprod}:

\begin{equation}\phantomsection\label{eq-fishprod}{
y=f(x,B(\tilde{w}))+\epsilon
}\end{equation}

The harvest technology function, \(f(x,B(\tilde{w}))\), is increasing
and concave in inputs and biomass. All other idiosyncratic shocks to
harvest that are not biological are captured in \(\epsilon\). Assuming
inputs are selected optimally conditional on the expected biomass, it is
clearer to show the effects of weather on harvest strictly in terms of
the random variable \(\tilde{w}\):

\begin{equation}\phantomsection\label{eq-fishprod2}{
y=f(B(\tilde{w}))+\epsilon
}\end{equation}

The weather and yield relationship, \(f(B(\tilde w))\) can be
empirically estimated by \(g(\tilde{w})\), with a model error term
\(\eta\)\footnote{Fisheries scientists might argue that you should first
  estimate \(b(\tilde w\)) and then estimate the relationship between
  yield and biomass. However, that would introduce an additional model
  error term and is not feasible for the case study. Market squid have
  such short lifespans that estimating their abundance is not
  worthwhile. Therefore the data does not exist in our context. Also
  insurance companies do not have access to the confidential data
  sources needed to build catch per unit effort and thus biomass
  estimates.}. Subbing the estimator into Equation~\ref{eq-fishprod2}
gives:

\begin{equation}\phantomsection\label{eq-g}{
y=g(\tilde{w})+\eta+\epsilon
}\end{equation}

Formally, basis risk is the combined stochastic effects of \(\eta\) and
\(\epsilon\). The model error, \(\eta\), arises from the inability to
perfectly estimate the weather to yield relationship. The idiosyncratic
error, \(\epsilon\), arises from all other shocks to yield that are not
related to weather. Well designed contracts minimize the negative
effects of design risk stemming from \(\eta\). For example, a perfectly
estimated weather to harvest relationship would minimize \(\eta\) to
zero.

The estimation of \(g(\tilde{w})\) is extremely difficult in fisheries.
However, the estimation does not need to be perfect or even
statistically significant to provide value to fishers. The model just
needs to capture enough of the weather to harvest relationship to
provide sufficient protection with the insurance contract.

Structural forms of all models are presented in detail in
Section~\ref{sec-methods}. For now, the general model output will be
represented as \(\hat{y}(w)\) stemming from estimator \(\hat{g}(w)\),
where \(w\) is the observed weather variables and \(\hat{y}\) is the
predicted harvest.

Index insurance contracts are designed as put options in
Equation~\ref{eq-gamma} that initiate payouts once a predicted models'
estimates, \(\hat{y}(w)\), fall below a critical yield threshold,
\(c*\mathbb{E}[y]\).

\begin{equation}\phantomsection\label{eq-gamma}{
I(w)=\max(c*\mathbb{E}[y]-\hat{y}(w),0)
}\end{equation}

Equation~\ref{eq-gamma} indicates fishers will receive the difference
between the long run average of catch and the model predicted outcome
from observed weather variables. The coverage level, \(c\), represents
the level of protection offered by the contract. I will test two
coverage levels, 70\% and 100\%, where the 70\% represents disaster
coverage that pays out only during extreme events, and 100\% indicating
any below average catch.

Premiums are calculated by determining the expected payout of the
contract times a premium loading factor, \(m\) as shown in equation
Equation~\ref{eq-prem}.

\begin{equation}\phantomsection\label{eq-prem}{
\rho(w)=\mathbb{E}[I(w)]m
}\end{equation}

I use burn rate analysis to calculate the expected payout in
Equation~\ref{eq-prem}. Burn rate analysis uses past realizations of
weather variables to create a distribution of payouts that would have
been observed if the policy was active over that time. The mean of the
payout distribution is used as the statistic for \(\mathbb{E}[I(w)]\).
Burn rate analysis is a common method used by actuaries to price index
insurance contracts and is suitable in this setting because of the short
time period of analysis.

\subsection{Utility evaluation}\label{utility-evaluation}

Evaluating the performance of index insurance contracts through a
utility framework provides more accurate measures for the impact of
basis risk on fishers willingness to pay for insurance (Conradt \emph{et
al.} 2015; Kenduiywo \emph{et al.} 2021). Extremely risk averse
policyholders will avoid insurance all together with any level of basis
risk as the downside risk of not receiving payouts when necessary
outweighs the benefits of having insurance (Clarke 2016). Therefore,
assessing contracts based on utility will provide more information on
fisher welfare especially in extreme events than simply comparing the
incidence of false negatives.

I calculate expected utility with, \(u_i\) and without insurance,
\(u_{ni}\), in the testing set as:

\begin{equation}\phantomsection\label{eq-ut}{
\begin{aligned}
u_i&=\frac{1}{n_{test}}\sum_{t}^{n_{test}}u(y_t+I_t(w_t)-\rho_t) \\
u_{ni}&=\frac{1}{n_{test}}\sum_{t}^{n_{test}}u(y_t)
\end{aligned}
}\end{equation}

Where \(y_t\) is the observed harvest per fisher in year \(t\),
\(I_t(w_t)\) is the payout from the insurance contract based on weather
variables \(w_t\), and \(\rho_t\) is the premium paid for insurance. The
utility function, \(u()\), is an increasing, concave function with risk
aversion. I use a negative exponential utility function with constant
absolute risk aversion in the main results of the paper. Constant
absolute risk aversion is preferred in this case as it remains defined
in the event of negative net income arising from large premiums and low
harvest. Other utility specifications with constant relative risk
aversion are included in the appendix. The expected utility with
insurance, \(u_i\), includes the payouts from the insurance contract
minus the premiums paid. The expected utility without insurance,
\(u_{ni}\), is simply the utility from the catch. The number of
observations in the test set is \(n_{test}\).

I determine whether fishers are better off with insurance by calculating
the percent change in utility from having insurance in
Equation~\ref{eq-urr}:

\begin{equation}\phantomsection\label{eq-urr}{
u_{rr}=\frac{u_i-u_{ni}}{u_{ni}}
}\end{equation}

Positive values of \(u_{rr}\) indicate fishers are better off with
insurance, whereas negative values indicate fishers are worse off with
insurance.

I use two loading factors, \(m\) to determine the viability of insurance
contracts. As a base, I use actuarially fair premiums where the premium
equals the expected payout of the contract
i.e.~\(\rho=\mathbb{E}[I(w)]\). Second, I calculate the premium loading
factor that makes fishers indifferent between having insurance and not
having insurance where \(u_{rr}=0\) as \(m^*\). The premium loading
factor is the multiple of the expected payout that insurance companies
charge to cover administrative costs and profit. Varying \(m\) only
affects the premiums paid. The payout schedule is solely determined
empirically by the models. Values of \(m^*\) greater than 1 indicate
fishers are willing to pay more than the expected payout for insurance,
whereas values below 1 indicate subsidies will be needed to stimulate
demand.

The \(m^*\) that sets \(u_{rr}=0\) will be called the ``market'' premium
as it represents the upper bound an insurance company could charge for
insurance. I calculate the loss ratio, \(lr=\sum I(\omega)/\sum \rho\),
at both actuarially fair and market premiums to analyze the market
viability of a contract. Loss ratios above 1 indicate insurance
companies lost money on the contract, whereas loss ratios below 1
indicate insurance companies are profitable. If a contract at \(m^*\)
leads to \(lr<1\), then the contract is viable in a free market.
Insurers can use any \(m\) less than the market \(m^*\) and still
encourage fishers to purchase insurance.

\section{Methods}\label{sec-methods}

I use five models to predict yearly catch \(\hat{y}(w)\) that serves as
the trigger within Equation~\ref{eq-gamma}. Linear models are the base
model given its ubiquitous use in index insurance policies. Linear
models create single index contracts that use only weather variable. I
refer to all linear regression models as ``univariate'' to differentiate
between the multivariate models. The four multivariate models will be
LASSO regression, random forest, regularized random forest, and support
vector machine as predictor models.

I use a rolling window cross validation method to assess the out of
sample predictive quality of each model, evaluate utility with insurance
in the test set, and tune hyperparameters of the multivariate models
(Ovando \emph{et al.} 2022; Akselrud 2024). The data is split into an
initial training window from 1990-2013 to provide enough data to
accurately calibrate a model. From the trained model, a single
prediction is made with environmental data one year in advance. The root
mean squared error is calculated as well as the payout amount based on
Equation~\ref{eq-gamma} for that year. The premium for that year is
calculated using burn rate pricing on the training set. In the next
period, the training window rolls forward one year to include 1990-2014.
The model is retrained and a prediction is made for 2015. This procedure
continues until the end of the data set in 2023. Expected utility is the
calculated for the test set based on the payout schedule and premiums
calculated as in Equation~\ref{eq-ut}.

The rolling window validation strategy is well suited for time series
data where temporal autocorrelation may be present. I do not provide a
validation window for parameter tuning as the time series is short.
Withholding observations may lead to more overfitting whereas I can use
the test set to tune hyperparameters. Additionally, the rolling window
mimics policywriters' strategies where they update their actuarial
models each year to provide the most accurate pricing information.

\subsection{Linear Models}\label{linear-models}

Perfect regression coefficients mimic the optimal choice of scale in
index insurance contracts (Mahul 1999). Combined with the ease of
implementation, linear models on single weather indices are the most
common design choice for index insurance policies (Benso \emph{et al.}
2023). Regression models offer a basic starting place to consider the
viability of fishery index insurance.

Annual catch is regressed on each environmental variable individually
using Ordinary Least Squares (OLS) as shown in Equation~\ref{eq-reg}.

\begin{equation}\phantomsection\label{eq-reg}{
y_{t}=\beta_0+\beta w_{t}+\epsilon_t
}\end{equation}

Linear models are preferred in index insurance because they provide a
simple, transparent relationship between the weather variable and catch.
For example, it is possible to show the output of a contract built on
linear regression in terms of the weather variable directly in
Equation~\ref{eq-lmi}.

\begin{equation}\phantomsection\label{eq-lmi}{
I(w)=\max(\beta_1c*(\mathbb{E}[y]-\frac{\beta_0}{\beta_1})-w,0)
}\end{equation}

The trigger, \((\mathbb{E}[y]-\frac{\beta_0}{\beta_1})\) is now
expressed directly in units of the weather variable while still
interpreted as a yield threshold. This structure allows for clear
communication to fishers about payouts. For example, if sea surface
temperature is the index, then the trigger can be expressed as 12
degrees Celsius. Any realized temperature, \(w\), below 12 degrees
Celsius would initiate a payout.

\subsection{LASSO Regression}\label{lasso-regression}

Least Absolute Shrinkage and Selection Operator (LASSO) regression is a
popular regularization technique to assist model selection. It attempts
to minimize the residual sum of squared errors through OLS, but adds a
penalty constraint on the absolute sum of selected coefficient values
(Equation~\ref{eq-lasso}).

\begin{equation}\phantomsection\label{eq-lasso}{
\hat{\beta}^{lasso}=\arg\min_{\beta}\left\{\sum_{i=1}^{n}(y_i-\beta_0-\sum_{j=1}^{p}\omega_{ij}\beta_j)^2+\lambda\sum_{j=1}^{p}|\beta_j|\right\}
}\end{equation}

Where, \(y_i\) is catch, \(\beta\) the regression coefficients, \(n\),
the number of observations, \(p\) the number of weather variables, and
\(w\) the total collection of weather variables. The \(\lambda\) is the
penalty term that controls the amount of shrinkage. Models are trained
using the \texttt{glmnet} package in R. The \(\lambda\) that minimizes
RMSE is selected through a grid search method. This choice is to ensure
the most parsimonious model that still captures the most important
weather variables. Reducing the number of variables used to define a
trigger payout will simplify the contract substantially.

\subsection{Random Forests}\label{random-forests}

Random Forests are tree-based ensemble models that capture non-linear
interactions through recursive partitioning. They are less sensitive to
over fitting through the aggregation of many trees. Each tree is built
on a bootstrapped sample of the training data. At each split, a random
subset of predictor variables is chosen to determine the best split. The
final prediction is the average of all the trees in the forest. Random
forests have been shown to outperform traditional regression models in
fisheries applications (Ovando \emph{et al.} 2022; Rahman \emph{et al.}
2022; Akselrud 2024).

The high dimmensionality of the data and few observations will make
random forests susceptible to overfitting. I address this high p low n
problem through feature selection and regularized random forests.
Certain weahter varialbes are likely to be highly correlated with each
other. I select the indices with the highest correlation to catch for
five categories of variables: Temperature, regional, upwelling, prey,
and paralarve abundance.

Regularizaed random forest provides an agonstic, data driven methodology
for feature selection. Regularized random forest computes permutation
improtance scores for each variable during out of bag testing. If the
increase in overall performance for a given variable is lower than a
specific threshold, it is removed from the model. Similar to LASSO, it
allows for an unbiased assessment of important variables. I can compare
whether my feature selection on expertise is accurate by examining the
variables selected by the regularized random forest.

I tune three parameters to optimize the random forest model. The number
of variables randomly sampled at each split, \(mtry\), the minimum
number of observations in a terminal node, \(nodesize\), and the maximum
depth of each tree. I manually construct a parameter grid to search for
the combination of parameters that minimize RMSE in the test set. Max
depth is set to small values, while \(nodesize\) has higher than default
thresholds to reduce overfitting bias. The small number of observations
in the data set (\(n=34\) years) necessitates more conservative
parameter choices. Models are built using the \texttt{ranger} package in
R.

Regularized random forest requires one additional tuning parameter along
with the three used in standard random forests. The regularization
parameter, \(\gamma_{rrf}\), controls the threshold a variable will be
added contigent on the permutation score. It is normalized from 0-1 with
1 acting as the most aggressive regularization. Models are built with
\texttt{grrf} package in R.

\subsection{Support Vector Machine}\label{support-vector-machine}

Support Vector Machine (SVM) constructs a decision boundary by
maximizing the margin between support vectors, the observations closest
to the separating hyperplane. I use a radial basis function kernel to
map the predictors into a higher-dimensional feature space, enabling the
algorithm to capture nonlinear patterns without explicitly defining
transformations of the input variables. Support Vector Machines work
well in time series data with low observations and high dimensionality
(\textbf{Sapankevych2009?}).

Model complexity is controlled by the regularization parameter, C, which
balances margin maximization against prediction error. A smaller C
allows for a wider margin, potentially misclassifying some training
points but enhancing generalization. A larger C aims to classify all
training points correctly, which may lead to overfitting. The kernel
coefficient, \(\gamma_{svm}\), defines the influence of individual
training examples. A small gamma implies a broader influence, leading to
smoother decision boundaries, while a large gamma focuses on nearby
points, capturing more intricate patterns but risking overfitting. I use
a grid search method to tune C and gamma to minimize RMSE in the test
set. Models are built using the \texttt{kernlab} package in R.

\subsection{Weather variables of
importance}\label{weather-variables-of-importance}

Machine learning algorithms are inherently ``black boxes'' that
sacrifice interpretability for predictive accuracy. Fishers will be less
likely to purchase complicated products that do not correspond to their
experiences. Extracting the relative contribution of weather variables
will assist translating products to fishers. Additionally, it can help
ground-truth the chosen variables with previous biological modelling.

I extract importance measures using permutation for all the multivariate
models. Permutation randomly modifies a variable's value then calculates
the change in the model performance. As models are updated in each
period, I will have \(n_{test}\) models each with their own permutation
scores. I examine the importance of each variable in a model over the
rolling training window.

\section{Market Squid Fishery}\label{sec-fish}

The California Market Squid Fishery is an ideal candidate to assess the
viability of index insurance due to its high economic value, simple
management system, excellent data, and ``corn-like'' biological
attributes.

Market squid is the second most valuable in California with an average
annual ex-vessel value of \$48 million since 2013. By landing weight, it
is the largest fishery in California with total catch exceeding half a
million tons since 2013. Fishing predominately occurs in the Santa
Barbara Channel during October and November. It is vulnerable to large
fluctuations in catch, which threatens livelihoods during disastrous
years. Index insurance would offer much needed stabilization to fishers
in this volatile fishery.

The fishery has been consistently managed since 2005 with a permit entry
system and a maximum annual catch limit of 118,000 metric tons. Prior to
2005, the fishery was open access with fishers only paying for a
licences and a landing tax. The number of licensed fishers dramatically
declined in the implementation of the permit system in 2005. However,
the number of fishers each year was recorded by the CDFW allowing for
the calculation of catch per fisher so that we may include landing
estimates prior to 2005 without biasing the data.

Fishery catch data is publicly available from the California Department
of Fish and Wildlife (CDFW) Marine Fisheries Data Explorer (MFDE). The
data contains catch records from 1980 to 2024 at the port-complex and
state level. Additionally, the data contains spatial records of catch
with specified fishing blocks that can be used to refine the location of
catch. I use this information to spatially match environmental variables
as shown in Figure~\ref{fig-cw} for sea surface temperature. This
procedure ensures that only the most relevant environmental data is used
to predict fishery harvest in the models.

\begin{figure}

\centering{

\includegraphics[width=7in,height=\textheight,keepaspectratio]{../data/fig/cw_catch.png}

}

\caption{\label{fig-cw}Demonstration of spatially matching catch data
with weather variables. Panel A shows sea surface temperature at a 5km
resolution. Panel B is the proproation of historical market squid catch
by fishing block. Panel C is the intersection of fishing grids and
weather variables. Sea surface temperature values are then averaged
within each block, and the weighted sum by percent of historical catch
is aggregated for each year.}

\end{figure}%

Market squid is a short lived species with a one year life cycle (Butler
and Fuller 1999). Sea surface temperature and upwelling primarily affect
recruitment leading to large fluctuations in abundance by the time
harvest occurs in the fall (Chasco \emph{et al.} 2022; Suca \emph{et
al.} 2022). Year to year variation does not appear to persist removing
dynamic considerations. This leads market squid to possess ``corn-like''
attributes akin to crops protected in agricultural index insurance.
Planting cycles each year are independent from prior weather outcomes. A
similar logic holds for market squid and allows the analysis to focus
solely on contemporaneous weather effects.

Other independent variables demonstrate correlations to squid abundance.
The Rockfish Recruitment and Ecosystem Assessment Survey (RREAS),
conducted by the Southwest Fishery Science Center, provides independent
surveys that collect tow samples that contain squid paralarvae and
krill. Each of these metrics could be used by insurance companies to
index fishery catch. Krill abundances are significantly correlated with
fall landings (Ralston \emph{et al.} 2018). Paralarvae measures were the
most important variable in random forest regressions to predict squid
fishery catch (Akselrud 2024). These variables are also indicative of
the potential to include non-weather variables that could serve as
triggers.

\section{Data}\label{sec-data}

The dataset contains 34 annual observations (1990-2023) of squid catch
per fisher and twelve weather variables. Catch histories extend to 1980,
but to ensure a balanced dataset I limit observations to 1990, which is
the earliest data available for squid and krill abundance in RREAS.

Fishery catch is publicly available through the California Department of
Fish and Wildlife Marine Data Explorer. State wide catch is divided by
the number of fishers to create metric tons of catch per fisher. Catch
per fisher is detrened using a robust method of moments estimator.
Detrending removes long term trends such as improved fishing technology,
and any potential effects of the management shift in 2005 so that
deviations from long run average are more directly attributable to
weather fluctuations. Observations are scaled relative to the last
observed value to avoid negative values.

Environmental data to define indices are summarized in
Table~\ref{tbl-env-sum}. Descriptions of the sources of data and
calculations are provided below.

Sea surface temperature data comes from the NOAA DHW that provides 5-km
resolution of monthly temperature from 1985 to 2023. The 5-km grids are
aggragated within the nearest California fishing block to provide an
annual time series.

The Habitat Compression Index measures the area extent of water below
average temperatures thresholds along the US West Coast (Schroeder
\emph{et al.} 2022). Habitat compression is a measure of the spatial
extent of cold water habitats that are important for fish species. The
index is broken down into four distinct oceanographic regions ranging
from 3.5 degrees to 5.5 degrees latitude in size with coverage out to
150 km offshore. I use the cumulative habitat compression index that
sums the index value in each month to provide a yearly time series of
habitat compression for each fishery. The cumulative index showed
stronger correlations with biological productivity measures than monthly
measures (Schroeder \emph{et al.} 2022)

I use five different sources to define the strength of upwelling.
Monthly observations of Coastal Upwelling Transport Index (CUTI) and
Biological Effective Upwelling Transport Index (BEUTI) are averaged over
the spring months of March, April, and May. Those months align most
closely with squid egg mixing and dispersion. Both indices create
measures of vertical movement in the mixed layer at 1 degree latitude
intervals extending 75 km along the entire US West Coast (Jacox \emph{et
al.} 2018). The closest layer to the surface was used in this analysis
as the correlation between surface index values and deeper index values
are high. CUTI examines the physical measures of wind, Ekman transport,
and cross-shore geostrophic transport to indicate the strength of
upwelling in a given month. BEUTI adds nitrate concentration in its
calculation to capture more biological effects of upwelling. Fishing
blocks are matched to the nearest 1 degree latitude interval to provide
a monthly time series of upwelling for each fishery. Seasonal strengths
of upwelling are captured by averaging CUTI and BEUTI within each
quarter of the year.

In addition to direct measurements of upwelling, I characterize three
upwelling events: Spring Transition Index, Relaxation, and Frequency
following methods from Suca \emph{et al.} (2022). Each variable was a
significant predictor in Suca \emph{et al.} (2022) recruitment models.
Spring transition is the date when the California Current transitions to
a more stable upwelling system after the turbulent winter months.
Mathematically, it is the first day of the year when the cumulative sum
of BEUTI reaches its minimum value. Relaxation is the count of the
number of days when CUTI exceeds 1.0 \(m^2s^{-1}\) followed by three or
more consecutive days of CUTI values \textless0.5 \(m^2s^{-1}\).
Frequency is the proportion of days when CUTI is measured between 2.0
and 0.5 \(m^2s^{-1}\). Frequency is indicative of moderate upwelling
events that stimulate primary production without pushing production away
from the coast.

Broader climate conditions have also been observed to impact squid
abundance. The Pacific Decadal Oscillation (PDO) and El Nino Southern
Oscillation (ENSO) have been linked to squid abundance (Perretti and
Sedarat 2016). ENSO has been used effectively in crop index insurance in
Peru. Larger climatic patterns may also be influential in squid
recruitment. PDO data is taken from the PDO ERSST V5 dataset. One
measure of ENSO data is taken from the multivariate ENSO Index Version 2
(MEI.v2). The last measure is the Oceanic Nino Index (ONI) that is the 3
month running mean of ERSST.v5 SST anomalies in the Nino 3.4 region.

Squid paralarvae and krill abundance are calculated from the Rockfish
Recruitment and Ecosystem Assessment Survey (RREAS). Midwater trawls
sample yearly at various stations along the California coast. I use
delta-generalized linear models to estimate log catch per unit effort of
paralarvae and krill for the whole state in each year\footnote{RREAS
  does not have consistent temporal or spatial coverage. From 1983-2003
  all surveys were done in a narrow window near Monterrey Bay. In 2004,
  more surveys locations were standardized into the Southern California
  Blight. Surveys were not conducted in 2014-2016 for the southern
  stations due to budget constraints. Using catch per unit effort for
  all stations provides a useful statewide proxy. Future use to make
  estimates more port specific would be valuable, but need to adjust for
  the data gaps}. Catch per unit effort serves as a proxy for prey and
squid paralarvae abundance.

\begin{table}

\caption{\label{tbl-env-sum}Summary of environmental variables used to
predict fishery catch.}

\centering{

\centering
\begin{tabular}{lllll}
\toprule
 & Variable Name & Temporal Resolution & Spatial Resolution & Source\\
\midrule
 & Sea Surface Temperature & Monthly & 5-km & NOAA Coral Bleaching Degree Heating Week\\
\cmidrule{2-5}
\multirow{-2}{*}{\raggedright\arraybackslash Temperature} & Cummulcative Habitat Compression Index & Annual & 1-degree latitude & Integrated Ecosystem Assessment\\
\cmidrule{1-5}
 & Spring BEUTI & Daily & 1-degree latitude & Jacox et al., 2018\\
\cmidrule{2-5}
 & Spring CUTI & Daily & 1-degree latitude & Jacox et al., 2018\\
\cmidrule{2-5}
 & Frequency & Daily & 1-degree latitude & Suca et al., 2022\\
\cmidrule{2-5}
 & Relaxation & Daily & 1-degree latitude & Suca et al., 2022\\
\cmidrule{2-5}
\multirow{-5}{*}{\raggedright\arraybackslash Upwelling} & Spring Transition Index & Daily & 1-degree latidue & Suca set al., 2022\\
\cmidrule{1-5}
 & Pacific Decadal Oscillation & Monthly & Regional & PDO ERSST V5\\
\cmidrule{2-5}
 & Oceanic Nino Index & Monthly & Regional & MEI.v2\\
\cmidrule{2-5}
\multirow{-3}{*}{\raggedright\arraybackslash Climate} & El Nino Southern Oscillation Index & Monthly & Regional & ONI\\
\cmidrule{1-5}
 & Squid Paralarvae & Annual & Statewide & RREAS\\
\cmidrule{2-5}
\multirow{-2}{*}{\raggedright\arraybackslash Biological} & Krill Abundance & Annual & Statewide & RREAS\\
\bottomrule
\end{tabular}

}

\end{table}%

\section{Results}\label{sec-results}

The multivariate models better predict weather-harvest dependency than
univariate models (Figure~\ref{fig-rmse}). However, overfitting is a
concern especially for the multivariate models. All multivariate models
had lower RMSE than every univariate linear model in the training set
(Panel A). In the test set (Panel B), multivariate models remained more
accurate as a group, but the discrepancy between the types of models was
much smaller. Support vector machine models had the lowest out of sample
RMSE for the multivariate models followed by random forest, and then
LASSO.

The best out of sample predictive model was frequency of upwelling
events with the lowest RMSE out of all models at 265 mt per fisher. Two
other groups of single weather indices performed well. Climate regional
measures, (PDO, ENSO, ONI) had lower error in both the test and training
sets than most other single variables. Krill abundance has the lowest
RMSE in the training set, but had the highest RMSE in the test set out
of all models.

\begin{figure}

\centering{

\includegraphics[width=5.5in,height=\textheight,keepaspectratio]{../data/fig/rmse.png}

}

\caption{\label{fig-rmse}The Root Mean Square Error of weather models
used to calculate insurance contracts. Panel A depicts the average RMSE
across all training set windows. Panel B is the total RMSE in the test
set using individual predictions from the rolling window. Model types
are separated into single index univariate models (blue) and
multivariate models (green)}

\end{figure}%

Examining fisher utility provides clearer measures of whether a model
sufficiently protects fishers from environmental risk. First, I present
the results of the contracts with actuarilly fair premiums, then select
only models that improved utility to calculate the market premium.

The majority of univariate models made fishers worse off as the fishers
experienced high degrees of downside risk where they did not receive
payouts when necessary (Figure~\ref{fig-ur-lr} Panel A). Only three
weather indices improved fisher utility: Upwelling frequency (35\%),
relaxation of upwelling (12\%), and krill abundance (8\%). However,
frequency and relaxation were highly unprofitable for insurance
companies as each had a loss ratio greater than 250\%
(Figure~\ref{fig-ur-lr} Panel B), which implies that insurance companies
would have to pay out \$2.50 for every \$1 of premium collected. The
upwelling indices paid large sums at times when fishers did not
experience loss. The large payouts act more as wealth transfers than
risk protection. Krill remained profitable for insurance companies with
a loss ratio of 55\% at actuarilly fair rates.

Every multivariate model improved fisher utility while reducing the
insurance company loss ratio (Figure~\ref{fig-ur-lr}) compared to the
univariate models. Contracts built with SVM and LASSO both improved
fisher utility by 37\% and 28\% respectively at actuarilly fair rates,
but each led to large losses for the insurance companies. Losses with
these two models were far less than those generated by frequency and
relaxation indices. Insurance companies remained solvent with both
random forest models, and fishers experienced 8-14\% gains in utility.

\begin{figure}

\centering{

\includegraphics[width=5.5in,height=\textheight,keepaspectratio]{../data/fig/act_fair.png}

}

\caption{\label{fig-ur-lr}Percent increase in utility with insurance
(Panel A) and the insurance loss ratio (Panel B) for each model using
actuarially fair premiums. Models that fall above the horizontal line in
Panel B are not profitable for insurance companies.}

\end{figure}%

Insurance companies would need to charge higher premiums to maintain
solvency for most of the models that fishers would be willing to pay
for. Figure~\ref{fig-urm-lrm} shows the market premium an insurance
company could charge at the upper limit of demand for each model in
Panel A. The ensuing loss ratio at that market premium is shown in Panel
B. Even at the upper limit of demand, insurance companies would not be
profitable with the upwelling models and LASSO. The loss ratios remain
above 1 implying insurance companies paid more out than they received at
the market premium. Fishers are most willing to pay for a contract with
the support vector machine as it has the highest premium. Insurance
companies would be willing to provide contracts at the market premiums
because the loss ratio is 0.75. Random forest contracts are also
profitable for insurance companies, though fishers are slightly less
willing to pay for them. The accuracy of random forest to avoid false
positives helps insurance companies avoid exessive payouts, allowing
insurers to remain profitable even at lower market premiums.

\begin{figure}

\centering{

\includegraphics[width=5.5in,height=\textheight,keepaspectratio]{../data/fig/market_prem.png}

}

\caption{\label{fig-urm-lrm}Market premiums rate (\(m^*\)) that make
fishers indifferent between having insurance and not having insurance
(Panel A). The corresponding loss ratio for contracts at the market rate
is in Panel B.}

\end{figure}%

Examining the payouts and premiums provides insight into how the
different contracts improve utility. Multivariate predictive models
initiate more frequent payouts, but payout larger quantities when
fishers experience low catch (Figure~\ref{fig-pay}). The flexibility of
the underlying predictive models drives the payout schedule. However,
with larger payouts, premiums must increase to compensate insurance
suppliers. Premiums with multivariate models are high and remain
consistent throughout the ten years of the test set even with updates
each year (Figure~\ref{fig-prem}). Contracts built with LASSO models
differ the most from the other multivariate models. LASSO contracts pay
out large sums in 2015 and 2023 when fishers experienced high catch.
Insurance companies suffer losses in those years limiting profits even
at the upper bound of market premiums.

The SVM and Random Forest contracts provide the largest utility gains
while offering competitive market premiums. Overall, both models
minimize downside risk for both fishers and insurance suppliers. There
are two key differences in the performance of both models that are more
easily observed when the net insurance payouts, payout minus the
premium, is compared to the realization of fisher catch as shown in
Figure~\ref{fig-inspay}. SVM contracts payout more during down years
compared to random forests. Larger payouts in periods of higher loss
provide large gains in utility. The higher payouts lead to the
corresponding higher premiums.

Random forests may be more accurate in identifying when to payout as the
amount of false positives and negatives favors random forests compared
to SVM. The notable exception is 2020 and 2021. Fishers experienced low
catch primarily due to COVID-19 closures and market disruptions.
COVID-19 was entirely independent of weather variables. Random forests
paid a small amount in 2020 (7 mt per fisher) as weather indicators
predicted a mild year in 2020. However, SVM paid out a larger amount (58
mt per fisher) in 2020. The large payout in 2020 was not necessary to
protect fishers from weather risk, but it did provide significant
utility gains as fishers were inadvertently protected from the COVID-19
disruptions.

Univariate models payout less frequently (Figure~\ref{fig-pay}). The
krill index paid out once during the lowest catch year in 2019. The
large one, time payout provides a significant gain in utility. Due to
fisher risk aversion, fishers are more willing to pay higher premiums to
protect against that single devastating loss (Figure~\ref{fig-prem}).
The upwelling indices pay more in later years. The frequency index pays
out every year from 2018-2023 with the largest payout in 2020 (197 mt
per fisher). These payout schedules by chance correspond to the COVID
disruptions. However, the payouts are not driven by weather risk, but
rather a coincidence that the upwelling indices predicted low catch
during those years. The consistent large payouts lead insurance
companies to significantly increase premiums as shown in
Figure~\ref{fig-prem}.

\begin{figure}

\centering{

\includegraphics[width=5.5in,height=\textheight,keepaspectratio]{../data/fig/pay_out.png}

}

\caption{\label{fig-pay}Payouts in the test set by each contract.
Multivariate models (left panel) have more flexible payout schedules
than utility improving univariate models (right panel). Individual
models are colored by model type.}

\end{figure}%

\begin{figure}

\centering{

\includegraphics[width=5.5in,height=\textheight,keepaspectratio]{../data/fig/prem_out.png}

}

\caption{\label{fig-prem}Premiums in the test set by each contract.
Multivariate models (left panel) charge higher premiums to compensate
for the more frequent payouts compared to univariate models (right
panel).}

\end{figure}%

\begin{figure}

\centering{

\includegraphics[width=5.5in,height=\textheight,keepaspectratio]{../data/fig/ins_pay.png}

}

\caption{\label{fig-inspay}Net insurance payouts in the testing set (red
points) from the SVM model and random forest (green points) compared to
the detrended catch per fisher data from the California Market Squid
Fishery (blue line and points). Insurance variables are in units of
harvest per fisher.}

\end{figure}%

Multivariate models are more accurate, improve fisher utility, and allow
insurance companies to earn a profit. However, the payout schedule is
determiend by complicated, non-linear interactions. This makes it
difficult to articulate which weather variable is driving the
determination of payouts to fishers. Variable importance measures
through permutation can help explain what weather variables possess the
largest influence in intitating payouts. Frequency, ENSO, and krill were
consistently the most important variables in the non-linear multivariate
models Figure~\ref{fig-varimp}. Frequency became more influential across
all models in later years. Sea surface temperature was mostly important
only within the random forests models.

I can evaluate ex-post whether my feature selection for random forests
\footnote{And to an extent SVM} was accurate by comparing the variables
of importance in the regularized random forest to those in the standard
random forest. The only three weather variables that were not
constrained to zero in the regularaized random forest, nor included in
the random forest, were Oceanic Nino Index, Pacific Decadal Ocisllation,
and the Habitat Compression Index. The MEIv2 El Nino index was
consistenly more important than the other two regional measures. Perhaps
more information could be captured with the inclusion of the other
regional climate variables, but each are highly correlated and capture
simialr patterns. Habitat Compression Index may have been included, but
had little importance compared to the other variables. In some years, it
was excluded from the model. Overall, the feature selection based on
correlation and expert knowledge was effective for both the random
forest and SVM, and did not exclude any of the primary weather
variables.

\begin{figure}

\centering{

\includegraphics[width=6in,height=\textheight,keepaspectratio]{../data/fig/vip_models.png}

}

\caption{\label{fig-varimp}Variable importance from permutation for
regularized random forest (left panel), random forest (middle panel),
and SVM (right panel) in each iteration of the training window.
Variables are colored and shaped by type.}

\end{figure}%

LASSO coefficients can be aggregated and divided by the absolute total
to get a measure of relative importance. Frequency is the most important
variable in LASSO in all years comprising about 75\% of the explanatory
power of the model Figure~\ref{fig-lassvip}. Spring CUTI was the second
most important variable in all years followed by MEIv2 El Nino.
Together, those three variables explained more the 90\% of the models
predictive power. A multiple linear regression contract with those three
variables could increase predictive power beyond a single index while
remaining clearly interpretable. However, given the lack of
profitability for the overall LASSO it is unlikely that such a model
would be market viable. The remaining environmental variables do not
provide much additional value to the model. Spring Transition Index,
PDO, CHCI, and SST are dropped from the model complete during some
training windows Zoomed panel of~\ref{fig-lassvip}.

\begin{figure}

\centering{

\includegraphics[width=6in,height=\textheight,keepaspectratio]{../data/fig/vip_lasso.png}

}

\caption{\label{fig-lassvip}Relative importance of LASSO regression
coefficients for each model trained in the training set. The right panel
focuses on variables with less than 10\% contribution.}

\end{figure}%

\section{Discussion}\label{sec-discussion}

Index insurance is feasible for the California Market Squid Fishery.
Contracts built with a single index may be the preferred option as they
are easy to implement and have clear, interpretable triggers, but the
inconsistency of payouts is undesirable for both fishers and insurers.
Out of twelve single index models, only three would have improved fisher
utility over the last decade. Fishery productivity is rarely driven by
individual weather patterns. It is unsurprising that using a single
measure would be provide sufficient risk protection.

Despite the preference for single index models, the performance of
single index models is lackluster even in agriculture. For example, in
the United States as part of the Rangeland, Pasture, Forage - Rainfall
Index (RPF-RI) program rainfall to production correlations may be as low
as 0.07 (Keller and Saitone 2022). Significant subsidies are needed to
support the RPF-RI due to the high levels of basis risk (Goodrich
\emph{et al.} 2019). Other agricultural IBI products do demonstrate
improved utility for farmers (Conradt \emph{et al.} 2015; Dalhaus
\emph{et al.} 2018; Kenduiywo \emph{et al.} 2021), but few studies
compare out of sample performance nor consider the supply side
incentives of insurance companies{[}Chen \emph{et al.} (2024);@. Without
assessing supply side viability may contribute to the low uptake of
index insurance in agriculture and high premiums (Binswanger-Mkhize
2012; Miranda and Farrin 2012). For example, Kenyan farmers gain utility
at actuarilly fair rates for maize rainfall contracts, but lose utility
at the commerical premiums (Jensen \emph{et al.} 2016). Single index
fishery contracts demonstrate similar patterns that could prohibit
effective deployment.

Machine learning models have been promoted as a response to the failures
of single index agricultural index insurance (Cesarini \emph{et al.}
2021). Few studies have used machine learning to define weather-yield
relationships, and as far as I am aware, no policy currently exists that
uses multivariate machine learning models. Fishery machine learning
models are market viable and provide large improvements to fisher
utility. Fishers are willing to consider complex models so long as the
basis risk remains low {[}Emlab Study{]}. The model variable importance
scores can help elucidate the ``black-box'' of machine learning models.
The variables that were most predictive as single indices remained
important in the multivariate models. Upwelling frequency, relaxation,
and krill abundance were key explanatory variables in all models. Even
though it is impossible to define a clear trigger for each variables in
these models, explaining that these intuitive variables contribute the
most to the determination of payouts will entrust fishers to the
contracts.

It has been proposed that models with lower basis risk should
intrinsically lower premiums. In this paper, the premiums fishers paid
for multivariate models was higher then single index models. This
phenomena originates from a unique form of data leakage that does not
interfere with model performance, but does influence premium
calculation. Nonlinear machine learning models accurately identify
extreme losses in the training sets. Payouts in the training set will be
much higher leading to more expensive premiums. Fishers are willing to
pay these higher premiums as the multivariate models accurately pay out
large sums in bad times. However, this does emphasize the need to avoid
overfitting as inaccurate out of sample predictions will lead fishers to
pay high premiums without appropriate compensation when needed. All
index insurance contracts that plan to use machine learning to identify
weather to yield relationships will be affected by this phenomena.

Hyperparameter tuning and feature selection are essential to reduce
overfitting. The rolling window strategy used in this paper demonstrates
an effective way to limit overfitting and data leakage. Feature
selection requires expertise in identifying potential indices that
affect catch.

The choice of fishery is an essential component to expand index
insurance. Each fishery is unique. Different weather variables will be
appropriate for fisheries outside of California Market Squid.
Diversifying the weather variable pool, spatially and directionally,
will allow insurance companies to diversify their risk more seamlessly
than fishers are able. For example, insurance companies could create
contracts for sea surface temperature in different ocean basins. The
risk of a marine heatwave to strike both the California Current
Ecosystem and the Gulf of Mexico is low. Additionally, species of fish
react differently to weather variables (Free \emph{et al.} 2019).
Offering contracts that cover different species will spread out risk for
insurance companies to avoid fund insolvency.

The selection criteria used in this paper to justify choosing California
Market Squid for analysis is not restrictive, but does demonstrate some
considerations that must be assessed when selecting fisheries for index
insurance.

Fast growing species with short life cycles are more likely to be
influenced by contemporaneous weather variables, and thus easier to
empirically identify weather to catch relationships. However, longer
lived species may also be insurable. The most important consideration is
whether an environmental variable that influences catch can be
identified. For example, Dungeness Crab have a four year life cycle.
Environmental variables during the megalopae stage affect adult crab
abundance and therefore harvest four years later (Shanks \emph{et al.}
2010; Magel \emph{et al.} 2020). If the variables are sufficiently
correlated with catch, then an insurance contract could designed based
on weather conditions four years prior. An interesting condsideration
for the contract would be the timing of the payout. Should the contract
pay out when the weather event occurs, or when fishers experience low
catch? Paying out when the weather event occurs would provide fishers
with more time to adjust their fishing strategies. However, the longer
the lag between the weather event and the payout, the more likely that
other confounding factors could influence catch.

Management provides the most interesting considerations. As described in
Section~\ref{sec-fish}, well managed fisheries often have the most
robust data available. However, the change in regulations influence
harvest more directly than weather stochasticity. Empirically
disentangling weather to catch relationships with changes in regulations
is difficult.

Well managed fisheries attempt to smooth biological risk by avoiding
boom and bust cycles. Fishers remain vulnerable to fluctuations in
management decisions such as quotas, season closures, or lost fishing
days. These actions more directly impact fisher livelihoods than
weather. Instead of designing contracts solely on weather variables, a
new category of indices directly related to fishery management could be
used. For example, quotas are set based on the underlying health of the
fishery. An index insurance contract could use the quota as the index.
This will provide a direct measure of lost catch for fishers while
remaining an independent measure. Currently, insurance companies have
demonstrated reticence on using management measures as they percieve
management actions to introduce human influence. However, quotas are set
based on rigorous scientific assessment outside the influence of fisher
actions. The index would remain independent of policyholder influence,
and is calculated in a transparent, objective manner.

Other triggers that originate from management could be used. Dungeness
crab fishers stated interest in insuring against the number of days the
fishing season is delayed by management. Preseason Salmon run strength
predictions could be another. The exact nature and suitability require
rigorous assessment and cooperation with management to ensure
independence.

This paper demonstrates fisheries index insurance is feasible in a
private market, but one-to-one application of techniques used in
agriculture is unlikely to be successful. Insurance companies currently
lack the expertise to navigate the complexitity of fisheries.
Partnerships between fisheries scientists, economists, and insurance
companies will be essential to design effective contracts. As fishing
communities face increasing environmental and economic uncertainty,
index insurance provides a promising tool to help fishers remain
financially resilient.

\newpage
\appendix
\renewcommand{\thefigure}{A\arabic{figure}}
\renewcommand{\thetable}{A\arabic{table}}
\setcounter{figure}{0}
\setcounter{table}{0}

\section{Appendix}\label{appendix}

\section*{References}\label{references}
\addcontentsline{toc}{section}{References}

\phantomsection\label{refs}
\begin{CSLReferences}{1}{0}
\bibitem[\citeproctext]{ref-AllenAkselrud2024}
Akselrud, C.I.A. (2024)
\href{https://doi.org/10.1016/j.fishres.2024.107161}{Random forest
regression models in ecology: Accounting for messy biological data and
producing predictions with uncertainty}. \emph{Fisheries Research}
\textbf{280}, 107161.

\bibitem[\citeproctext]{ref-Barnett2007}
Barnett, B.J. and Mahul, O. (2007)
\href{https://doi.org/10.1111/j.1467-8276.2007.01091.x}{Weather index
insurance for agriculture and rural areas in lower-income countries}.
\emph{American Journal of Agricultural Economics} \textbf{89},
1241--1247.

\bibitem[\citeproctext]{ref-Benso2023}
Benso, M.R., Gesualdo, G.C., Silva, R.F., et al. (2023)
\href{https://doi.org/10.5194/nhess-23-1335-2023}{Review article: Design
and evaluation of weather index insurance for multi-hazard resilience
and food insecurity}. \emph{Natural Hazards and Earth System Sciences}
\textbf{23}, 1335--1354.

\bibitem[\citeproctext]{ref-binswanger2012}
Binswanger-Mkhize, H.P. (2012)
\href{https://doi.org/10.1080/00220388.2011.625411}{Is there too much
hype about index-based agricultural insurance?} \emph{Journal of
Development Studies} \textbf{48}, 187--200.

\bibitem[\citeproctext]{ref-Butler1999}
Butler, J. and Fuller, D. (1999) BUTLER ET AL.: AGE AND GROWTH OF MARKET
SQUID OFF CALIFORNIA CalCOFl rep AGE AND GROWTH OF MARKET SQUID (LOLIGO
OPALESCENS) OFF CALIFORNIA DURING 1998.

\bibitem[\citeproctext]{ref-Cesarini2021}
Cesarini, L., Figueiredo, R., Monteleone, B. and Martina, M.L.V. (2021)
\href{https://doi.org/10.5194/nhess-21-2379-2021}{The potential of
machine learning for weather index insurance}. \emph{Natural Hazards and
Earth System Sciences} \textbf{21}, 2379--2405.

\bibitem[\citeproctext]{ref-Chasco2022}
Chasco, B.E., Hunsicker, M.E., Jacobson, K.C., Welch, O.T., Morgan,
C.A., Muhling, B.A. and Harding, J.A. (2022)
\href{https://doi.org/10.1002/mcf2.10190}{Evidence of temperature-driven
shifts in market squid doryteuthis opalescens densities and distribution
in the california current ecosystem}. \emph{Marine and Coastal
Fisheries} \textbf{14}, 1--13.

\bibitem[\citeproctext]{ref-Chen2024}
Chen, Z., Lu, Y., Zhang, J. and Zhu, W. (2024)
\href{https://doi.org/10.1287/mnsc.2023.4902}{Managing weather risk with
a neural network-based index insurance}. \emph{Management Science}
\textbf{70}, 4306--4327.

\bibitem[\citeproctext]{ref-Clarke2016}
Clarke, D.J. (2016) \href{https://doi.org/10.1257/mic.20140103}{A theory
of rational demand for index insurance}. \emph{Journal: Microeconomics}
\textbf{8}, 283--306.

\bibitem[\citeproctext]{ref-Clement2018}
Clement, K.Y., Botzen, W.J.W., Brouwer, R. and Aerts, J.C.J.H. (2018)
\href{https://doi.org/10.1016/j.ijdrr.2018.01.001}{A global review of
the impact of basis risk on the functioning of and demand for index
insurance}. \emph{International Journal of Disaster Risk Reduction}
\textbf{28}, 845--853.

\bibitem[\citeproctext]{ref-Conradt2015}
Conradt, S., Finger, R. and Spörri, M. (2015)
\href{https://doi.org/10.1016/j.crm.2015.06.003}{Flexible weather
index-based insurance design}. \emph{Climate Risk Management}
\textbf{10}, 106--117.

\bibitem[\citeproctext]{ref-Correia2021}
Correia, H.E. (2021)
\href{https://doi.org/10.1038/s41598-021-89398-8}{Semiparametric model
selection for identification of environmental covariates related to
adult groundfish catches and weights}. \emph{Scientific Reports}
\textbf{11}, 1--14.

\bibitem[\citeproctext]{ref-Costello2008}
Costello, C. and Polasky, S. (2008)
\href{https://doi.org/10.1016/j.jeem.2008.03.001}{Optimal harvesting of
stochastic spatial resources}. \emph{Journal of Environmental Economics
and Management} \textbf{56}, 1--18.

\bibitem[\citeproctext]{ref-Dalhaus2018}
Dalhaus, T., Musshoff, O. and Finger, R. (2018)
\href{https://doi.org/10.1038/s41598-017-18656-5}{Phenology information
contributes to reduce temporal basis risk in agricultural weather index
insurance}. \emph{Scientific Reports} \textbf{8}, 1--10.

\bibitem[\citeproctext]{ref-Feng2019}
Feng, P., Wang, B., Liu, D.L., Waters, C. and Yu, Q. (2019)
\href{https://doi.org/10.1016/j.agrformet.2019.05.018}{Incorporating
machine learning with biophysical model can improve the evaluation of
climate extremes impacts on wheat yield in south-eastern australia}.
\emph{Agricultural and Forest Meteorology} \textbf{275}, 100--113.

\bibitem[\citeproctext]{ref-Free2019}
Free, C.M., Thorson, J.T., Plinsky, M.L., Oken, K.L., Wiedenmann, J. and
Jensen, O.P. (2019)
\href{https://doi.org/10.1126/science.aax5721}{Impacts of historical
warming on marine fisheries production}. \emph{Science} \textbf{365},
979--983.

\bibitem[\citeproctext]{ref-Goodrich2019}
Goodrich, B., Yu, J. and Vandeveer, M. (2019)
\href{https://doi.org/10.1057/s41288-019-00149-3}{Participation patterns
of the rainfall index insurance for pasture, rangeland and forage
programme}. \emph{The Geneva Papers on Risk and Insurance - Issues and
Practice} \textbf{45}, 29--51.

\bibitem[\citeproctext]{ref-Herrmann2004}
Herrmann, M., Greenberg, J., Hamel, C. and Geier, H. (2004)
\href{https://doi.org/10.1577/M02-086.1}{Extending federal crop
insurance programs to commercial fisheries: The case of bristol bay,
alaska, sockeye salmon}. 352--366 pp.

\bibitem[\citeproctext]{ref-Hilborn2003}
Hilborn, R., Quinn, T.P., Schindler, D.E. and Rogers, D.E. (2003)
\href{https://doi.org/10.1073/pnas.103727410}{Biocomplexity and
fisheries sustainability}. \emph{PNAS} \textbf{100}, 6564--6568.

\bibitem[\citeproctext]{ref-Hobday2025}
Hobday, A.J., Little, L.R., Watson, J.R. and Spillman, C.M. (2025)
\href{https://doi.org/10.1007/s11160-025-09920-3}{Parametric insurance
for climate adaptation in fisheries and aquaculture}. \emph{Reviews in
Fish Biology and Fisheries} \textbf{35}, 175--185.

\bibitem[\citeproctext]{ref-Holland2020}
Holland, D.S. and Leonard, J. (2020)
\href{https://doi.org/10.1016/j.hal.2020.101904}{Is a delay a disaster?
Economic impacts of the delay of the california dungeness crab fishery
due to a harmful algal bloom}. \emph{Harmful Algae} \textbf{98}.

\bibitem[\citeproctext]{ref-Jacox2018}
Jacox, M.G., Edwards, C.A., Hazen, E.L. and Bograd, S.J. (2018)
\href{https://doi.org/10.1029/2018JC014187}{Coastal upwelling revisited:
Ekman, bakun, and improved upwelling indices for the u.s. West coast}.
\emph{Journal of Geophysical Research: Oceans} \textbf{123}, 7332--7350.

\bibitem[\citeproctext]{ref-Jardine2020}
Jardine, S.L., Fisher, M.C., Moore, S.K. and Samhouri, J.F. (2020)
\href{https://doi.org/10.1016/j.ecolecon.2020.106691}{Inequality in the
economic impacts from climate shocks in fisheries: The case of harmful
algal blooms}. \emph{Ecological Economics} \textbf{176}.

\bibitem[\citeproctext]{ref-Jensen2016}
Jensen, N.D., Barrett, C.B. and Mude, A.G. (2016)
\href{https://doi.org/10.1093/ajae/aaw046}{Index insurance quality and
basis risk: Evidence from northern kenya}. \emph{American Journal of
Agricultural Economics} \textbf{98}, 1450--1469.

\bibitem[\citeproctext]{ref-Jensen2019}
Jensen, N., Stoeffler, Q., Fava, F., et al. (2019)
\href{https://doi.org/10.1016/J.ECOLECON.2019.04.014}{Does the design
matter? Comparing satellite-based indices for insuring pastoralists
against drought}. \emph{Ecological Economics} \textbf{162}, 59--73.

\bibitem[\citeproctext]{ref-Kasperski2013}
Kasperski, S. and Holland, D.S. (2013)
\href{https://doi.org/10.1073/pnas.1212278110}{Income diversification
and risk for fishermen}. \emph{Proceedings of the National Academy of
Sciences of the United States of America} \textbf{110}, 2076--2081.

\bibitem[\citeproctext]{ref-Keller2022}
Keller, J.B. and Saitone, T.L. (2022)
\href{https://doi.org/10.1111/ajae.12282}{Basis risk in the pasture,
rangeland, and forage insurance program: Evidence from california}.
\emph{Amer. J. Agr. Econ} \textbf{104}, 1203--1223.

\bibitem[\citeproctext]{ref-Kenduiywo2021}
Kenduiywo, B.K., Carter, M.R., Ghosh, A. and Hijmans, R.J. (2021)
\href{https://doi.org/10.1371/journal.pone.0258215}{Evaluating the
quality of remote sensing products for agricultural index insurance}.
\emph{PLoS ONE} \textbf{16}.

\bibitem[\citeproctext]{ref-vanklompenburg2020}
Klompenburg, T. van, Kassahun, A. and Catal, C. (2020)
\href{https://doi.org/10.1016/j.compag.2020.105709}{Crop yield
prediction using machine learning: A systematic literature review}.
\emph{Computers and Electronics in Agriculture} \textbf{177}.

\bibitem[\citeproctext]{ref-Lehodey2006}
Lehodey, P., Alheit, J., Barange, M., et al. (2006) Climate variability,
fish, and fisheries.

\bibitem[\citeproctext]{ref-Magel2020}
Magel, C.L., Lee, E.M.J., Strawn, A.M., Swieca, K. and Jensen, A.D.
(2020) \href{https://doi.org/10.3389/fmars.2020.00401}{Connecting crabs,
currents, and coastal communities: Examining the impacts of changing
ocean conditions on the distribution of u.s. West coast dungeness crab
commercial catch}. \emph{Frontiers in Marine Science} \textbf{7}, 1--16.

\bibitem[\citeproctext]{ref-Mahul1999}
Mahul, O. (1999) \href{https://doi.org/10.2307/1244451}{Optimum area
yield crop insurance}. \emph{American Journal of Agricultural Economics}
\textbf{81}, 75--82.

\bibitem[\citeproctext]{ref-Miranda2012}
Miranda, M.J. and Farrin, K. (2012)
\href{https://doi.org/10.1093/aepp/pps031}{Index insurance for
developing countries}. \emph{Applied Economic Perspectives and Policy}
\textbf{34}, 391--427.

\bibitem[\citeproctext]{ref-Mumford2009}
Mumford, J.D., Leach, A.W., Levontin, P. and Kell, L.T. (2009)
\href{https://doi.org/10.1093/icesjms/fsp100}{Insurance mechanisms to
mediate economic risks in marine fisheries}. \emph{ICES Journal of
Marine Science} \textbf{66}, 950--959.

\bibitem[\citeproctext]{ref-Ovando2022}
Ovando, D., Cunningham, C., Kuriyama, P., Boatright, C. and Hilborn, R.
(2022) \href{https://doi.org/10.1139/cjfas-2021-0287}{Improving
forecasts of sockeye salmon (oncorhynchus nerka) with parametric and
nonparametric models}. \emph{Canadian Journal of Fisheries and Aquatic
Sciences} \textbf{79}, 1198--1210.

\bibitem[\citeproctext]{ref-Perretti2016}
Perretti, C.T. and Sedarat, M. (2016)
\href{https://doi.org/10.1111/fog.12167}{The influence of the el niño
southern oscillation on paralarval market squid (doryteuthis
opalescens)}. \emph{Fisheries Oceanography} \textbf{25}, 491--499.

\bibitem[\citeproctext]{ref-Privitera2020}
Privitera-Johnson, K.M. and Punt, A.E. (2020)
\href{https://doi.org/10.1016/j.fishres.2020.105503}{A review of
approaches to quantifying uncertainty in fisheries stock assessments}.
\emph{Fisheries Research} \textbf{226}, 105503.

\bibitem[\citeproctext]{ref-Rahman2022}
Rahman, L.F., Marufuzzaman, M., Alam, L., Bari, M.A., Sumaila, U.R. and
Sidek, L.M. (2022)
\href{https://doi.org/10.1007/s40009-022-01110-0}{Application of machine
learning to investigate the impact of climatic variables on marine fish
landings}. \emph{National Academy Science Letters} \textbf{45},
245--248.

\bibitem[\citeproctext]{ref-Ralston2018}
Ralston, S., Dorval, E., Ryley, L., Sakuma, K.M. and Field, J.C. (2018)
\href{https://doi.org/10.1016/j.fishres.2017.11.009}{Predicting market
squid (doryteuthis opalescens) landings from pre-recruit abundance}.
\emph{Fisheries Research} \textbf{199}, 12--18.

\bibitem[\citeproctext]{ref-Sainsbury2019}
Sainsbury, N.C., Turner, R.A., Townhill, B.L., Mangi, S.C. and Pinnegar,
J.K. (2019) \href{https://doi.org/10.1038/s41558-019-0645-z}{The
challenges of extending climate risk insurance to fisheries}.
\emph{Nature Climate Change} \textbf{9}, 896--897.

\bibitem[\citeproctext]{ref-Schmidt2022}
Schmidt, L., Odening, M., Schlanstein, J. and Ritter, M. (2022)
\href{https://doi.org/10.1016/J.AGSY.2021.103345}{Exploring the
weather-yield nexus with artificial neural networks}. \emph{Agricultural
Systems} \textbf{196}.

\bibitem[\citeproctext]{ref-Schroeder2022}
Schroeder, I.D., Santora, J.A., Mantua, N., et al. (2022)
\href{https://doi.org/10.1016/j.ecolind.2022.109520}{Habitat compression
indices for monitoring ocean conditions and ecosystem impacts within
coastal upwelling systems}. \emph{Ecological Indicators} \textbf{144},
109520.

\bibitem[\citeproctext]{ref-Sethi2010}
Sethi, S.A. (2010)
\href{https://doi.org/10.1111/j.1467-2979.2010.00363.x}{Risk management
for fisheries}. \emph{Fish and Fisheries} \textbf{11}, 341--365.

\bibitem[\citeproctext]{ref-Shanks2010}
Shanks, A., Roegner, G.C. and Miller, J. (2010) Using megalopae
abundance to predict future commercial catches of dungeness crabs
(cancer magister) in oregon. \emph{California Cooperative Oceanic
Fisheries Investigations Reports} \textbf{51}, 106--118.

\bibitem[\citeproctext]{ref-Smith2023}
Smith, K.E., Burrows, M.T., Hobday, A.J., et al. (2023)
\href{https://doi.org/10.1146/annurev-marine-032122-121437}{Biological
impacts of marine heatwaves}. \emph{Annual Review of Marine Science}
\textbf{15}, 1--27.

\bibitem[\citeproctext]{ref-deSouza2016}
Souza, E.N.D., Boerder, K., Matwin, S. and Worm, B. (2016)
\href{https://doi.org/10.1371/journal.pone.0158248}{Improving fishing
pattern detection from satellite AIS using data mining and machine
learning}. \emph{PLoS ONE} \textbf{11}.

\bibitem[\citeproctext]{ref-Spicka2013}
Spicka, J. and Hnilica, J. (2013)
\href{https://doi.org/10.1155/2013/146036}{A methodical approach to
design and valuation of weather derivatives in agriculture}.
\emph{Advances in Meteorology} \textbf{2013}.

\bibitem[\citeproctext]{ref-Suca2022}
Suca, J.J., Santora, J.A., Field, J.C., et al. (2022)
\href{https://doi.org/10.1093/icesjms/fsac186}{Temperature and upwelling
dynamics drive market squid (doryteuthis opalescens) distribution and
abundance in the california current}. \emph{ICES Journal of Marine
Science} \textbf{79}, 2489--2509.

\bibitem[\citeproctext]{ref-Sumaila2021}
Sumaila, U.R., Walsh, M., Hoareau, K., et al. (2021)
\href{https://doi.org/10.1038/s41467-021-23168-y}{Financing a
sustainable ocean economy}. \emph{Nature Communications} \textbf{12}.

\bibitem[\citeproctext]{ref-Szuwalski2023}
Szuwalski, C.S., Aydin, K., Fedewa, E.J., Garber-Yonts, B. and Litzow,
M.A. (2023) \href{https://doi.org/10.1126/SCIENCE.ADF6035}{The collapse
of eastern bering sea snow crab}. \emph{Science} \textbf{382}, 306--310.

\bibitem[\citeproctext]{ref-villasenor2024}
Villaseñor-Derbez, J.C., Arafeh-Dalmau, N. and Micheli, F. (2024)
\href{https://doi.org/10.1038/s43247-024-01696-x}{Past and future
impacts of marine heatwaves on small-scale fisheries in baja california,
mexico}. \emph{Communications Earth and Environment} \textbf{5}.

\bibitem[\citeproctext]{ref-Watson2023}
Watson, J.R., Spillman, C.M., Little, L.R., Hobday, A.J. and Levin, P.S.
(2023) \href{https://doi.org/10.1093/icesjms/fsad175}{Enhancing the
resilience of blue foods to climate shocks using insurance}. \emph{ICES
Journal of Marine Science} \textbf{80}, 2457--2469.

\end{CSLReferences}




\end{document}
