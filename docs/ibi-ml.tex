% Options for packages loaded elsewhere
\PassOptionsToPackage{unicode}{hyperref}
\PassOptionsToPackage{hyphens}{url}
\PassOptionsToPackage{dvipsnames,svgnames,x11names}{xcolor}
%
\documentclass[
  letterpaper,
  DIV=11,
  numbers=noendperiod]{scrartcl}

\usepackage{amsmath,amssymb}
\usepackage{iftex}
\ifPDFTeX
  \usepackage[T1]{fontenc}
  \usepackage[utf8]{inputenc}
  \usepackage{textcomp} % provide euro and other symbols
\else % if luatex or xetex
  \usepackage{unicode-math}
  \defaultfontfeatures{Scale=MatchLowercase}
  \defaultfontfeatures[\rmfamily]{Ligatures=TeX,Scale=1}
\fi
\usepackage{lmodern}
\ifPDFTeX\else  
    % xetex/luatex font selection
\fi
% Use upquote if available, for straight quotes in verbatim environments
\IfFileExists{upquote.sty}{\usepackage{upquote}}{}
\IfFileExists{microtype.sty}{% use microtype if available
  \usepackage[]{microtype}
  \UseMicrotypeSet[protrusion]{basicmath} % disable protrusion for tt fonts
}{}
\makeatletter
\@ifundefined{KOMAClassName}{% if non-KOMA class
  \IfFileExists{parskip.sty}{%
    \usepackage{parskip}
  }{% else
    \setlength{\parindent}{0pt}
    \setlength{\parskip}{6pt plus 2pt minus 1pt}}
}{% if KOMA class
  \KOMAoptions{parskip=half}}
\makeatother
\usepackage{xcolor}
\setlength{\emergencystretch}{3em} % prevent overfull lines
\setcounter{secnumdepth}{5}
% Make \paragraph and \subparagraph free-standing
\ifx\paragraph\undefined\else
  \let\oldparagraph\paragraph
  \renewcommand{\paragraph}[1]{\oldparagraph{#1}\mbox{}}
\fi
\ifx\subparagraph\undefined\else
  \let\oldsubparagraph\subparagraph
  \renewcommand{\subparagraph}[1]{\oldsubparagraph{#1}\mbox{}}
\fi


\providecommand{\tightlist}{%
  \setlength{\itemsep}{0pt}\setlength{\parskip}{0pt}}\usepackage{longtable,booktabs,array}
\usepackage{calc} % for calculating minipage widths
% Correct order of tables after \paragraph or \subparagraph
\usepackage{etoolbox}
\makeatletter
\patchcmd\longtable{\par}{\if@noskipsec\mbox{}\fi\par}{}{}
\makeatother
% Allow footnotes in longtable head/foot
\IfFileExists{footnotehyper.sty}{\usepackage{footnotehyper}}{\usepackage{footnote}}
\makesavenoteenv{longtable}
\usepackage{graphicx}
\makeatletter
\def\maxwidth{\ifdim\Gin@nat@width>\linewidth\linewidth\else\Gin@nat@width\fi}
\def\maxheight{\ifdim\Gin@nat@height>\textheight\textheight\else\Gin@nat@height\fi}
\makeatother
% Scale images if necessary, so that they will not overflow the page
% margins by default, and it is still possible to overwrite the defaults
% using explicit options in \includegraphics[width, height, ...]{}
\setkeys{Gin}{width=\maxwidth,height=\maxheight,keepaspectratio}
% Set default figure placement to htbp
\makeatletter
\def\fps@figure{htbp}
\makeatother
\newlength{\cslhangindent}
\setlength{\cslhangindent}{1.5em}
\newlength{\csllabelwidth}
\setlength{\csllabelwidth}{3em}
\newlength{\cslentryspacingunit} % times entry-spacing
\setlength{\cslentryspacingunit}{\parskip}
\newenvironment{CSLReferences}[2] % #1 hanging-ident, #2 entry spacing
 {% don't indent paragraphs
  \setlength{\parindent}{0pt}
  % turn on hanging indent if param 1 is 1
  \ifodd #1
  \let\oldpar\par
  \def\par{\hangindent=\cslhangindent\oldpar}
  \fi
  % set entry spacing
  \setlength{\parskip}{#2\cslentryspacingunit}
 }%
 {}
\usepackage{calc}
\newcommand{\CSLBlock}[1]{#1\hfill\break}
\newcommand{\CSLLeftMargin}[1]{\parbox[t]{\csllabelwidth}{#1}}
\newcommand{\CSLRightInline}[1]{\parbox[t]{\linewidth - \csllabelwidth}{#1}\break}
\newcommand{\CSLIndent}[1]{\hspace{\cslhangindent}#1}

\addtokomafont{disposition}{\rmfamily}
\KOMAoption{captions}{tableheading}
\makeatletter
\makeatother
\makeatletter
\makeatother
\makeatletter
\@ifpackageloaded{caption}{}{\usepackage{caption}}
\AtBeginDocument{%
\ifdefined\contentsname
  \renewcommand*\contentsname{Table of contents}
\else
  \newcommand\contentsname{Table of contents}
\fi
\ifdefined\listfigurename
  \renewcommand*\listfigurename{List of Figures}
\else
  \newcommand\listfigurename{List of Figures}
\fi
\ifdefined\listtablename
  \renewcommand*\listtablename{List of Tables}
\else
  \newcommand\listtablename{List of Tables}
\fi
\ifdefined\figurename
  \renewcommand*\figurename{Figure}
\else
  \newcommand\figurename{Figure}
\fi
\ifdefined\tablename
  \renewcommand*\tablename{Table}
\else
  \newcommand\tablename{Table}
\fi
}
\@ifpackageloaded{float}{}{\usepackage{float}}
\floatstyle{ruled}
\@ifundefined{c@chapter}{\newfloat{codelisting}{h}{lop}}{\newfloat{codelisting}{h}{lop}[chapter]}
\floatname{codelisting}{Listing}
\newcommand*\listoflistings{\listof{codelisting}{List of Listings}}
\makeatother
\makeatletter
\@ifpackageloaded{caption}{}{\usepackage{caption}}
\@ifpackageloaded{subcaption}{}{\usepackage{subcaption}}
\makeatother
\makeatletter
\@ifpackageloaded{tcolorbox}{}{\usepackage[skins,breakable]{tcolorbox}}
\makeatother
\makeatletter
\@ifundefined{shadecolor}{\definecolor{shadecolor}{rgb}{.97, .97, .97}}
\makeatother
\makeatletter
\makeatother
\makeatletter
\makeatother
\ifLuaTeX
  \usepackage{selnolig}  % disable illegal ligatures
\fi
\IfFileExists{bookmark.sty}{\usepackage{bookmark}}{\usepackage{hyperref}}
\IfFileExists{xurl.sty}{\usepackage{xurl}}{} % add URL line breaks if available
\urlstyle{same} % disable monospaced font for URLs
\hypersetup{
  pdftitle={Finding suitable weather indices for novel fisheries index insurance using machine learning},
  pdfauthor={Nathaniel Grimes},
  pdfkeywords={Index Insurance, Fisheries, Machine Learning},
  colorlinks=true,
  linkcolor={blue},
  filecolor={Maroon},
  citecolor={Blue},
  urlcolor={Blue},
  pdfcreator={LaTeX via pandoc}}

\title{Finding suitable weather indices for novel fisheries index
insurance using machine learning}
\author{Nathaniel Grimes}
\date{2024-08-08}

\begin{document}
\maketitle
\begin{abstract}
Index insurance is a financial tool gaining traction for application in
fisheries. It will cover fishers losses under extreme weather events
that impact fishery productivity. This is the first assessment to
determine the feasibility of such programs and whether suitable indices
exist.
\end{abstract}
\ifdefined\Shaded\renewenvironment{Shaded}{\begin{tcolorbox}[sharp corners, frame hidden, interior hidden, boxrule=0pt, breakable, enhanced, borderline west={3pt}{0pt}{shadecolor}]}{\end{tcolorbox}}\fi

\renewcommand*\contentsname{Table of contents}
{
\hypersetup{linkcolor=}
\setcounter{tocdepth}{3}
\tableofcontents
}
\hypertarget{introduction}{%
\section{Introduction}\label{introduction}}

Predicting fishery output from weather variables is notoriously
difficult. It is widely established that climate and weather affect
fishing populations (Lehodey \emph{et al.} 2006), but most stock
assessment models use little to no year to year environmental data
(Privitera-Johnson and Punt 2020). Variations in environmental
conditions are now the leading cause for fishery closures and disaster
relief payouts in the United States (Bellquist \emph{et al.} 2021).
Disaster declarations are becoming more frequent straining a slow,
inequitable system (Holland and Leonard 2020; Jardine \emph{et al.}
2020). Calls for new financial tools to alleviate fisher income shocks
have grown (Mumford \emph{et al.} 2009; Sethi 2010).

Index insurance has risen as a prime candidate tool to protect fishing
communities during disasters (\textbf{Watson2023?}). Index insurance is
a financial product that pays out when an independently verified index,
such as rainfall or temperature, falls below a predetermined threshold.
The index is chosen to be highly correlated with the asset being
insured. Index insurance has been successful in agriculture, but has not
been widely adopted in fisheries. The main reason is that suitable
indices for fisheries are not well understood. This study aims to
identify suitable indices for fisheries index insurance using machine
learning.

Effective index insurance policies require clear connections between the
index and policyholder assets. Otherwise basis risk is introduced. Basis
risk is the probability that policyholders experience a harmful shock to
their income, but the index does not trigger. Basis risk lowers demand
for index insurance and remains a significant roadblock in setting up
new programs (Binswanger-Mkhize 2012; Clarke 2016).

Three strategies are often used to mitigate basis risk and stimulate
uptake. First, government subsidies directly mask the ineffectiveness of
some triggers. The United States Risk Management Agency's Rainfall index
insurance for pasture, rangeland, and forage (RI-PRF) allows farmers to
select grids of forage for cattle ranching and protect against low
rainfall in 2 month intervals (e.g.~Jan-Feb). Subsidies encourage
farmers to buy products through reducing the premium paid by up to 60\%.
In Nebraska and Kansas from 2013 to 2017, the program had negative
returns overall, but farmers had net positive income strictly due to the
subsidies (Goodrich \emph{et al.} 2019). Basis risk in Nebraska and
Kansas introduced a 26\% probability of insurance not paying out when
damages were suffered (Yu \emph{et al.} 2019). The same program in
California found basis risk reaching up to 46\%, which could be driven
by weak correlations (\(r \in[0.071,0.417]\)) between indices and forage
production (Keller and Saitone 2022).

Contract design can mitigate basis risk through providing more options
so that individuals can better select policies that protect them.
Policyholders choose lower trigger levels when correlations between
between index and asset are low (Lichtenberg and Iglesias 2022). Lower
trigger levels correspond to protection against more catastrophic
shocks. Increased contract flexibility reduces basis risk by only small
amounts. Yu \emph{et al.} (2019) found that more flexible contracts
could account for only 5-9\% of basis risk. Farms in Kansas closer to
weather stations had better predictive impacts of rain on yield (Yu
\emph{et al.} 2019). The best way to reduce basis risk is to define
accurate correlations of index to loss (Jensen \emph{et al.} 2019;
\textbf{Carter2015?}).

\hypertarget{data}{%
\section{Data}\label{data}}

This study attempts to cover breadth, not depth in possible indices.
Each fishery has unique ecological characteristics that interact with
environmental variables in different and non-linear ways. By studying a
wide collection of fisheries and environmental variables we can uncover
the potential feasibility of index insurance for fisheries. Landings and
revenue data comes from the West Coast Fish data package. It is a
reconstruction of California Fish and Wildlife Department catch data
combined with PacFin receipts for Washington and Oregon.

We select California fisheries with a minimum of 30 years of consecutive
catch records at both the state and port-complex level. Unclassified
catch records are dropped i.e.~``Other Sharks'' and similar categories.
We spatially refine catch histories using the California CDFW fishing
blocks records.

\hypertarget{methods}{%
\section*{Methods}\label{methods}}
\addcontentsline{toc}{section}{Methods}

\hypertarget{refs}{}
\begin{CSLReferences}{1}{0}
\leavevmode\vadjust pre{\hypertarget{ref-Bellquist2021}{}}%
Bellquist, L., Saccomanno, V., Semmens, B.X., Gleason, M. and Wilson, J.
(2021) \href{https://doi.org/10.7717/peerj.11186}{The rise in climate
change-induced federal fishery disasters in the united states}.
\emph{PeerJ} \textbf{9}.

\leavevmode\vadjust pre{\hypertarget{ref-binswanger2012}{}}%
Binswanger-Mkhize, H.P. (2012)
\href{https://doi.org/10.1080/00220388.2011.625411}{Is there too much
hype about index-based agricultural insurance?} \emph{Journal of
Development Studies} \textbf{48}, 187--200.

\leavevmode\vadjust pre{\hypertarget{ref-Clarke2016}{}}%
Clarke, D.J. (2016) \href{https://doi.org/10.1257/mic.20140103}{A theory
of rational demand for index insurance}. \emph{Journal: Microeconomics}
\textbf{8}, 283--306.

\leavevmode\vadjust pre{\hypertarget{ref-Goodrich2019}{}}%
Goodrich, B., Yu, J. and Vandeveer, M. (2019)
\href{https://doi.org/10.1057/s41288-019-00149-3}{Participation patterns
of the rainfall index insurance for pasture, rangeland and forage
programme}. \emph{The Geneva Papers on Risk and Insurance - Issues and
Practice} \textbf{45}, 29--51.

\leavevmode\vadjust pre{\hypertarget{ref-Holland2020}{}}%
Holland, D.S. and Leonard, J. (2020)
\href{https://doi.org/10.1016/j.hal.2020.101904}{Is a delay a disaster?
Economic impacts of the delay of the california dungeness crab fishery
due to a harmful algal bloom}. \emph{Harmful Algae} \textbf{98}.

\leavevmode\vadjust pre{\hypertarget{ref-Jardine2020}{}}%
Jardine, S.L., Fisher, M.C., Moore, S.K. and Samhouri, J.F. (2020)
\href{https://doi.org/10.1016/j.ecolecon.2020.106691}{Inequality in the
economic impacts from climate shocks in fisheries: The case of harmful
algal blooms}. \emph{Ecological Economics} \textbf{176}.

\leavevmode\vadjust pre{\hypertarget{ref-Jensen2019}{}}%
Jensen, N., Stoeffler, Q., Fava, F., et al. (2019)
\href{https://doi.org/10.1016/J.ECOLECON.2019.04.014}{Does the design
matter? Comparing satellite-based indices for insuring pastoralists
against drought}. \emph{Ecological Economics} \textbf{162}, 59--73.

\leavevmode\vadjust pre{\hypertarget{ref-Keller2022}{}}%
Keller, J.B. and Saitone, T.L. (2022)
\href{https://doi.org/10.1111/ajae.12282}{Basis risk in the pasture,
rangeland, and forage insurance program: Evidence from california}.
\emph{Amer. J. Agr. Econ} \textbf{104}, 1203--1223.

\leavevmode\vadjust pre{\hypertarget{ref-Lehodey2006}{}}%
Lehodey, P., Alheit, J., Barange, M., et al. (2006) Climate variability,
fish, and fisheries.

\leavevmode\vadjust pre{\hypertarget{ref-Lichtenberg2022}{}}%
Lichtenberg, E. and Iglesias, E. (2022)
\href{https://doi.org/10.1016/j.jdeveco.2022.102883}{Index insurance and
basis risk: A reconsideration}. \emph{Journal of Development Economics}
\textbf{158}.

\leavevmode\vadjust pre{\hypertarget{ref-Mumford2009}{}}%
Mumford, J.D., Leach, A.W., Levontin, P. and Kell, L.T. (2009)
\href{https://doi.org/10.1093/icesjms/fsp100}{Insurance mechanisms to
mediate economic risks in marine fisheries}. \emph{ICES Journal of
Marine Science} \textbf{66}, 950--959.

\leavevmode\vadjust pre{\hypertarget{ref-Privitera2020}{}}%
Privitera-Johnson, K.M. and Punt, A.E. (2020)
\href{https://doi.org/10.1016/j.fishres.2020.105503}{A review of
approaches to quantifying uncertainty in fisheries stock assessments}.
\emph{Fisheries Research} \textbf{226}, 105503.

\leavevmode\vadjust pre{\hypertarget{ref-Sethi2010}{}}%
Sethi, S.A. (2010)
\href{https://doi.org/10.1111/j.1467-2979.2010.00363.x}{Risk management
for fisheries}. \emph{Fish and Fisheries} \textbf{11}, 341--365.

\leavevmode\vadjust pre{\hypertarget{ref-Yu2019}{}}%
Yu, J., Vandeveer, M., Volesky, J.D. and Harmoney, K. (2019)
\href{https://doi.org/10.22004/ag.econ.281319}{Estimating the basis risk
of rainfall index insurance for pasture, rangeland, and forage}.
\emph{Journal of Agricultural and Resource Economics} \textbf{44},
179--193.

\end{CSLReferences}



\end{document}
