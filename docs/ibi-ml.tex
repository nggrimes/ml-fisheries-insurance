% Options for packages loaded elsewhere
\PassOptionsToPackage{unicode}{hyperref}
\PassOptionsToPackage{hyphens}{url}
\PassOptionsToPackage{dvipsnames,svgnames,x11names}{xcolor}
%
\documentclass[
  letterpaper,
  DIV=11,
  numbers=noendperiod]{scrartcl}

\usepackage{amsmath,amssymb}
\usepackage{iftex}
\ifPDFTeX
  \usepackage[T1]{fontenc}
  \usepackage[utf8]{inputenc}
  \usepackage{textcomp} % provide euro and other symbols
\else % if luatex or xetex
  \usepackage{unicode-math}
  \defaultfontfeatures{Scale=MatchLowercase}
  \defaultfontfeatures[\rmfamily]{Ligatures=TeX,Scale=1}
\fi
\usepackage{lmodern}
\ifPDFTeX\else  
    % xetex/luatex font selection
\fi
% Use upquote if available, for straight quotes in verbatim environments
\IfFileExists{upquote.sty}{\usepackage{upquote}}{}
\IfFileExists{microtype.sty}{% use microtype if available
  \usepackage[]{microtype}
  \UseMicrotypeSet[protrusion]{basicmath} % disable protrusion for tt fonts
}{}
\makeatletter
\@ifundefined{KOMAClassName}{% if non-KOMA class
  \IfFileExists{parskip.sty}{%
    \usepackage{parskip}
  }{% else
    \setlength{\parindent}{0pt}
    \setlength{\parskip}{6pt plus 2pt minus 1pt}}
}{% if KOMA class
  \KOMAoptions{parskip=half}}
\makeatother
\usepackage{xcolor}
\setlength{\emergencystretch}{3em} % prevent overfull lines
\setcounter{secnumdepth}{5}
% Make \paragraph and \subparagraph free-standing
\ifx\paragraph\undefined\else
  \let\oldparagraph\paragraph
  \renewcommand{\paragraph}[1]{\oldparagraph{#1}\mbox{}}
\fi
\ifx\subparagraph\undefined\else
  \let\oldsubparagraph\subparagraph
  \renewcommand{\subparagraph}[1]{\oldsubparagraph{#1}\mbox{}}
\fi


\providecommand{\tightlist}{%
  \setlength{\itemsep}{0pt}\setlength{\parskip}{0pt}}\usepackage{longtable,booktabs,array}
\usepackage{calc} % for calculating minipage widths
% Correct order of tables after \paragraph or \subparagraph
\usepackage{etoolbox}
\makeatletter
\patchcmd\longtable{\par}{\if@noskipsec\mbox{}\fi\par}{}{}
\makeatother
% Allow footnotes in longtable head/foot
\IfFileExists{footnotehyper.sty}{\usepackage{footnotehyper}}{\usepackage{footnote}}
\makesavenoteenv{longtable}
\usepackage{graphicx}
\makeatletter
\def\maxwidth{\ifdim\Gin@nat@width>\linewidth\linewidth\else\Gin@nat@width\fi}
\def\maxheight{\ifdim\Gin@nat@height>\textheight\textheight\else\Gin@nat@height\fi}
\makeatother
% Scale images if necessary, so that they will not overflow the page
% margins by default, and it is still possible to overwrite the defaults
% using explicit options in \includegraphics[width, height, ...]{}
\setkeys{Gin}{width=\maxwidth,height=\maxheight,keepaspectratio}
% Set default figure placement to htbp
\makeatletter
\def\fps@figure{htbp}
\makeatother
\newlength{\cslhangindent}
\setlength{\cslhangindent}{1.5em}
\newlength{\csllabelwidth}
\setlength{\csllabelwidth}{3em}
\newlength{\cslentryspacingunit} % times entry-spacing
\setlength{\cslentryspacingunit}{\parskip}
\newenvironment{CSLReferences}[2] % #1 hanging-ident, #2 entry spacing
 {% don't indent paragraphs
  \setlength{\parindent}{0pt}
  % turn on hanging indent if param 1 is 1
  \ifodd #1
  \let\oldpar\par
  \def\par{\hangindent=\cslhangindent\oldpar}
  \fi
  % set entry spacing
  \setlength{\parskip}{#2\cslentryspacingunit}
 }%
 {}
\usepackage{calc}
\newcommand{\CSLBlock}[1]{#1\hfill\break}
\newcommand{\CSLLeftMargin}[1]{\parbox[t]{\csllabelwidth}{#1}}
\newcommand{\CSLRightInline}[1]{\parbox[t]{\linewidth - \csllabelwidth}{#1}\break}
\newcommand{\CSLIndent}[1]{\hspace{\cslhangindent}#1}

\addtokomafont{disposition}{\rmfamily}
\KOMAoption{captions}{tableheading}
\makeatletter
\makeatother
\makeatletter
\makeatother
\makeatletter
\@ifpackageloaded{caption}{}{\usepackage{caption}}
\AtBeginDocument{%
\ifdefined\contentsname
  \renewcommand*\contentsname{Table of contents}
\else
  \newcommand\contentsname{Table of contents}
\fi
\ifdefined\listfigurename
  \renewcommand*\listfigurename{List of Figures}
\else
  \newcommand\listfigurename{List of Figures}
\fi
\ifdefined\listtablename
  \renewcommand*\listtablename{List of Tables}
\else
  \newcommand\listtablename{List of Tables}
\fi
\ifdefined\figurename
  \renewcommand*\figurename{Figure}
\else
  \newcommand\figurename{Figure}
\fi
\ifdefined\tablename
  \renewcommand*\tablename{Table}
\else
  \newcommand\tablename{Table}
\fi
}
\@ifpackageloaded{float}{}{\usepackage{float}}
\floatstyle{ruled}
\@ifundefined{c@chapter}{\newfloat{codelisting}{h}{lop}}{\newfloat{codelisting}{h}{lop}[chapter]}
\floatname{codelisting}{Listing}
\newcommand*\listoflistings{\listof{codelisting}{List of Listings}}
\makeatother
\makeatletter
\@ifpackageloaded{caption}{}{\usepackage{caption}}
\@ifpackageloaded{subcaption}{}{\usepackage{subcaption}}
\makeatother
\makeatletter
\@ifpackageloaded{tcolorbox}{}{\usepackage[skins,breakable]{tcolorbox}}
\makeatother
\makeatletter
\@ifundefined{shadecolor}{\definecolor{shadecolor}{rgb}{.97, .97, .97}}
\makeatother
\makeatletter
\makeatother
\makeatletter
\makeatother
\ifLuaTeX
  \usepackage{selnolig}  % disable illegal ligatures
\fi
\IfFileExists{bookmark.sty}{\usepackage{bookmark}}{\usepackage{hyperref}}
\IfFileExists{xurl.sty}{\usepackage{xurl}}{} % add URL line breaks if available
\urlstyle{same} % disable monospaced font for URLs
\hypersetup{
  pdftitle={Finding suitable weather indices for novel fisheries index insurance using machine learning},
  pdfauthor={Nathaniel Grimes},
  pdfkeywords={Index Insurance, Fisheries, Machine Learning},
  colorlinks=true,
  linkcolor={blue},
  filecolor={Maroon},
  citecolor={Blue},
  urlcolor={Blue},
  pdfcreator={LaTeX via pandoc}}

\title{Finding suitable weather indices for novel fisheries index
insurance using machine learning}
\author{Nathaniel Grimes}
\date{2024-10-07}

\begin{document}
\maketitle
\begin{abstract}
Index insurance is a financial tool gaining traction for application in
fisheries. It will cover fishers losses under extreme weather events
that impact fishery productivity. This is the first assessment to
determine the feasibility of such programs and whether suitable indices
exist.
\end{abstract}
\ifdefined\Shaded\renewenvironment{Shaded}{\begin{tcolorbox}[sharp corners, borderline west={3pt}{0pt}{shadecolor}, boxrule=0pt, enhanced, interior hidden, frame hidden, breakable]}{\end{tcolorbox}}\fi

\renewcommand*\contentsname{Table of contents}
{
\hypersetup{linkcolor=}
\setcounter{tocdepth}{3}
\tableofcontents
}
\hypertarget{introduction}{%
\section{Introduction}\label{introduction}}

Predicting fishery output from weather variables is notoriously
difficult. It is widely established that climate and weather affect
fishing populations (Lehodey \emph{et al.} 2006), but most stock
assessment models use little to no year to year environmental data
(Privitera-Johnson and Punt 2020). Variations in environmental
conditions are now the leading cause of fishery closures and disaster
relief payouts in the United States (Bellquist \emph{et al.} 2021).
Disaster declarations are becoming more frequent straining a slow,
inequitable system (Holland and Leonard 2020; Jardine \emph{et al.}
2020). Calls for new financial tools to alleviate fisher income shocks
have grown (Mumford \emph{et al.} 2009; Sethi 2010).

Index insurance has risen as a prime candidate tool to protect fishing
communities during disasters (Watson \emph{et al.} 2023). Index
insurance is a financial product that pays out when an independently
verified index, such as rainfall or temperature, falls below a
predetermined threshold. The index is chosen to be highly correlated
with the asset being insured. However, fisheries are notoriously
difficult to establish clear, concise weather impacts on productivity.
Fisheries are dynamic systems where weather shocks can leave residual
impacts years after the initial shock (\textbf{Hilborn2003?}).
Individual fisheries can cover enormous areas in ocean basins. The
expansive spatial coverage of fisheries makes it unclear where and how
much specific weather variables have on biological abundance. In
addition, if weather impacts have been observed, they are most likely
highly non-linear adding further complexity. The greatest impediment to
the development of fishery insurance policies is reconciling these
challenges to find suitable indices that can predict fishery
productivity (Watson \emph{et al.} 2023).

Recent expansions in oceanic remote sensing has led to a wealth of new
environmental indices that could be used to predict fishery
productivity. Fishery data collection continues to improve with better
reporting systems with longer and more detailed catch histories. This
study aims to leverage these improved data sources to create suitable
indices for fisheries index insurance using machine learning.

The difficulty in modelling fishery productivity with environmental
indices leads to basis risk. Formally, basis risk is the probability
that policyholders experience a harmful shock to their income, but the
index does not trigger. Basis risk lowers demand for index insurance and
remains a significant roadblock in setting up new programs
(Binswanger-Mkhize 2012; Clarke 2016). Designing indices with stronger
correlations to fishery losses is the most effective way to reduce basis
risk (Jensen \emph{et al.} 2019).

It is impossible to completely eliminate basis risk. In agriculture,
every policy possess some level of basis risk. Well designed policies
that capture up to 90\% of the income variation in K

Three strategies are often used to mitigate basis risk and stimulate
uptake. First, government subsidies directly mask the ineffectiveness of
some triggers. The United States Risk Management Agency's Rainfall index
insurance for pasture, rangeland, and forage (RI-PRF) allows farmers to
select grids of forage for cattle ranching and protect against low
rainfall in 2 month intervals (e.g.~Jan-Feb). Subsidies encourage
farmers to buy products through reducing the premium paid by up to 60\%.
In Nebraska and Kansas from 2013 to 2017, the program had negative
returns overall, but farmers had net positive income strictly due to the
subsidies (Goodrich \emph{et al.} 2019). Basis risk in Nebraska and
Kansas introduced a 26\% probability of insurance not paying out when
damages were suffered (Yu \emph{et al.} 2019). The same program in
California found basis risk reaching up to 46\%, which could be driven
by weak correlations (\(r \in[0.071,0.417]\)) between indices and forage
production (Keller and Saitone 2022).

Contract design can mitigate basis risk through providing more options
so that individuals can better select policies that protect them.
Policyholders choose lower trigger levels when correlations between
between index and asset are low (Lichtenberg and Iglesias 2022). Lower
trigger levels correspond to protection against more catastrophic
shocks. Increased contract flexibility reduces basis risk by only small
amounts. Yu \emph{et al.} (2019) found that more flexible contracts
could account for only 5-9\% of basis risk. Farms in Kansas closer to
weather stations had better predictive impacts of rain on yield (Yu
\emph{et al.} 2019).

Machine learning has exploded as a new tool to define new and better
indices in agriculture index insurance Schmidt \emph{et al.} (2022).
Machine learning models excel in index insurance because indemnity
contracts only need predictive relationships. Fishery stock assessments
build complex models with biological foundations to accurately inform
management of future fish stocks. Index insurance can look retroactively
at data to uncover relationships and test out of sample predictive
quality. The application of machine learning is growing in fisheries to
answer a variety of data needs. Ensemble models built through
combinations of random forests, boosted trees, and dynamic linear models
improved Bristol Bay sockeye salmon forecasts by 15\% compared to a
standard lagged regression model (Ovando \emph{et al.} 2022).
Environmental variables of importance to groundfish populations in
Alaska were uncovered using single index varying coefficient models
regularized with LASSO (\textbf{Correia2018?}). Random Forests models
better predict fish catch in Indonesia than traditional linear models
(Rahman \emph{et al.} 2022). The expected non-linear interactions of
weather and fishery productivity merit.

This study will provide the first comprehensive examination of the
necessary features of weather indices for fisheries index insurance.

We will use machine learning to uncover the most important weather
variables for predicting fishery productivity. We will then use these
variables to create a novel index insurance model for fisheries. We will
test the model on a wide array of fisheries in California to determine
the feasibility of index insurance for fisheries.

\hypertarget{sec-model}{%
\section{Insurance Model}\label{sec-model}}

Insurance contracts are specified by calculating payout functions
(\(I(\omega)\)) based on independently measured weather variables.
Neural networks have been used to provide non-linear payoff schedules
that better reduce basis risk (Chen \emph{et al.} 2024). While it has
been shown that linear payoff functions inherently lead to basis risk
and therefore lower demand (Clarke 2016; Jensen \emph{et al.} 2016), we
maintain their use to preserve measures of interpretability that are
clearer for a first analysis of fishery index insurance.

Payouts will be issued when prediction models predict negative
deviations from long run average value. The three prediction models
(\(k\in\{\text{LR,LA,RF}\}\) are a linear regression (\(\text{LR}\)), a
LASSO regression (\(\text{LA}\)), and a random forest (\(\text{RF}\)).

\begin{equation}\protect\hypertarget{eq-payout}{}{
I(\omega)=\max(0,(\bar{\pi}-\hat\pi_t^k(\omega)) \cdot l)
}\label{eq-payout}\end{equation}

Where \(k\) is the prediction model, \(l\) is the level of coverage,
\(\hat\pi_t^k(w)\) is the predicted fishing variable from \(\omega\)
weather variables, and \(\bar{\pi}_{x}\) the long run average of the
fishing variable. The premium is calculated as the expected value of the
payout function times the premium loading factor (\(m\)).

\begin{equation}\protect\hypertarget{eq-premium}{}{
\rho(\omega)=\mathbb{E}[I(\omega)]m
}\label{eq-premium}\end{equation}

Utility measures offer the most insightful evaluation of index insurance
policies (Kenduiywo \emph{et al.} 2021). It captures value added for
policyholders, not just measures of payout frequency as other measures
of basis risk. Constant absolute risk aversion allows more consistent
comparison for different levels of wealth. Expected utility for a given
fishery is the average utility over all years in the sample for any
variable of interest \(\pi_t\). Fishers are allowed to choose insurance
coverage levels \(l\) to ensure feasible contracts.

\[
\begin{aligned}
\mathbb{E}[U_{b}]&=\frac{1}{n}\sum_{t}^{T}\frac{1-e^{-a\pi_t}}{a} &\text{No Insurance}\\
\mathbb{E}[U_{i}]&=\max_{l}\frac{1}{n}\sum_{t}^{T}\frac{1-e^{(-a(\pi_t+I(\omega,l)-\rho(w))}}{a} &\text{Insurance}\\
U_{r}&=\frac{\mathbb{E}[U_i]-\mathbb{E}[U_b]}{\mathbb{E}[U_b]}\cdot100 &\text{Percent Change in Utility}\\
\end{aligned}
\]

We will compare the percent change in fisher utility with insurance
(\(U_i\)) versus without insurance (\(U_{b}\)) for each prediction
method. The variable of interest \(\pi_t\), will be fishing revenue,
landings, and catch per fisher to test what measures of fishery
productivity are most suitable for index insurance.

No data on fishery insurance suppliers exists to create a market
equilibrium. We iteratively vary the premium loading \(m\in[1,2]\) to
create a range of coverage values fishers will be willing to pay for a
given \(m\). Then, the amount of coverage purchased times the premium
loading factor approximately equals the expected revenue an insurance
company would receive. Insurance companies could then examine their own
administrative and legal costs to determine whether the feasible
contract is profitable.

\hypertarget{sec-data}{%
\section{Data}\label{sec-data}}

This study attempts to cover breadth, not depth in possible indices.
Each fishery has unique ecological characteristics that interact with
environmental variables in different and non-linear ways. By studying a
wide collection of fisheries and environmental variables we can uncover
the potential feasibility of index insurance for fisheries holistically,
and then further refine measures with ecologically sound models in the
future.

\hypertarget{fishery-data}{%
\subsection{Fishery Data}\label{fishery-data}}

Landings revenue, and participation data comes from the West Coast Fish
data package (Free \emph{et al.} 2022). It is a reconstruction of
California Fish and Wildlife Department catch data combined with PacFin
receipts for Washington and Oregon. The last three years of data are
updated from the CDFW Marine Fisheries Data Explorer (MFDE). Names are
matched to each species within the West Coast Fish data package.

We select California fisheries with a minimum of 30 years of
consecutive, non-confidential catch records at both the state and
port-complex level. Unclassified catch records are dropped i.e.~``Other
Sharks'' and similar categories. Fisheries with an average revenue
greater than \$100,000 at the state and \$75,000 at port-complex level
are analyzed. Twenty four fisheries at the state level and 49 fisheries
at the port complex level meet these criteria. These fisheries contain
the most economically important fisheries in California and their mean
values are shown in \textbf{?@tbl-fish-sum}.

Fisheries have complex spatial dynamics. Agriculture has clear,
quantifiable impacts of weather in grids that are well suited for index
insurance. Drought on a single farm directly leads to crop loss for that
farm. Whether there is sufficient spatial coverage to identify impacts
down to an individual farm remains a challenge in agriculture (Dalhaus
and Finger 2016; Leppert \emph{et al.} 2021; Stigler and Lobell 2024).
Fish and fishers can move thousand of miles in a given year, thus more
consideration must be given to the location of weather impacts in
fisheries. We spatially refine catch histories using the California CDFW
fishing blocks records from the MFDE Data Explorer. Summarized catch
histories of all landed fish within each block provide an average
representation of effort for a given fishery. Spatial catch history is
measured at both the state and port-complex level. The spatial location
refines the location of environmental variables. Local weather is more
likely to affect fishery productivity and catch than observations
thousands of miles away.

\hypertarget{environmental-data}{%
\subsection{Environmental Data}\label{environmental-data}}

Fisheries are highly sensitive to marine heatwaves and water
temperature. Sea surface temperature is a natural variable to first
consider in fisheries index insurance. Sea surface temperature data
comes from the NOAA DHW data set that provides 5-km resolution of
monthly temperature from 1985 to 2023. The 5-km grids are averaged
within the nearest California fishing block to provide an annual time
series of temperature for each fishery. Temperature is lagged from 1 to
3 years prior to account for residual impacts that carry over due to
fishery biological dynamics.

Upwelling provides vital nutrients to stimulate primary productivity.
The coast of California is a highly productive ecosystem due to its
patterns of upwelling (Chelton \emph{et al.} 1982; Huyer 1983). We
capture upwelling through monthly observations of Coastal Upwelling
Transport Index (CUTI) and Biological Effective Upwelling Transport
Idnex (BEUTI). Both indices create measures of vertical movement in the
mixed layer at 1 degree latitude intervals extending 75 km along the
entire US West Coast (Jacox \emph{et al.} 2018). The closest layer to
the surface was used in this analysis as the correlation between surface
index values and deeper index values are high. CUTI examines the
physical measures of wind, ekman transport, and cross-shore geostrophic
transport to indicate the strength of upwelling in a given month. BEUTI
adds nitrate concentration in its calculation to capture more biological
effects of upwelling. Fishing blocks are matched to the nearest 1 degree
latitude interval to provide a monthly time series of upwelling for each
fishery. Seasonal strengths of upwelling are captured by averaging CUTI
and BEUTI within each quarter of the year. Spring upwelling in early
March and April are espeically important to a wide array of fish
species. Yearly average and amplitude values (the difference between
minimum observed upwelling and maximum) are also calculated. These
indices are the most temporally limited datasets in this analysis, only
extending from 1988 to 2023.

The Habitat Compression Index measures the area extent of water below
average temperatures thresholds along the US West Coast (Schroeder
\emph{et al.} 2022). Habitat compression is a measure of the spatial
extent of cold water habitats that are important for fish species. The
index is broken down into four distinct oceangraphic regions ranging
from 3.5 degrees to 5.5 degrees lattitude in size with coverage out to
150 km offshore. We use the cumulative habitat compression index that
sums the index value in each month to provide a yearly time series of
habitat compression for each fishery. The cumulative index showed
stronger correlations with biological productivity measures than monthly
measures (Schroeder \emph{et al.} 2022)

The final environmental variables are the Pacific Decadal Oscillation
(PDO) and the El Nino Southern Oscillation (ENSO). Both indices are well
known to affect marine ecosystems and fisheries. Both indices are
averaged over a given year. PDO data is taken from the PDO ERSST V5, and
ENSO data is taken from the multivariate ENSO Index Version 2 (MEI.v2).

Summary statistics for the environmental data are presented in
\textbf{?@tbl-env-sum}. In total, 73 fisheries with 35 years of catch
data are matched to 20 spatial matched environmental variables with
annual coverage from 1988 to 2023.

\hypertarget{sec-methods}{%
\section{Methods}\label{sec-methods}}

\hypertarget{sec-results}{%
\section{Results}\label{sec-results}}

\hypertarget{sec-discussion}{%
\section*{Discussion}\label{sec-discussion}}
\addcontentsline{toc}{section}{Discussion}

\hypertarget{refs}{}
\begin{CSLReferences}{1}{0}
\leavevmode\vadjust pre{\hypertarget{ref-Bellquist2021}{}}%
Bellquist, L., Saccomanno, V., Semmens, B.X., Gleason, M. and Wilson, J.
(2021) \href{https://doi.org/10.7717/peerj.11186}{The rise in climate
change-induced federal fishery disasters in the united states}.
\emph{PeerJ} \textbf{9}.

\leavevmode\vadjust pre{\hypertarget{ref-binswanger2012}{}}%
Binswanger-Mkhize, H.P. (2012)
\href{https://doi.org/10.1080/00220388.2011.625411}{Is there too much
hype about index-based agricultural insurance?} \emph{Journal of
Development Studies} \textbf{48}, 187--200.

\leavevmode\vadjust pre{\hypertarget{ref-Cesarini2021}{}}%
Cesarini, L., Figueiredo, R., Monteleone, B. and Martina, M.L.V. (2021)
\href{https://doi.org/10.5194/nhess-21-2379-2021}{The potential of
machine learning for weather index insurance}. \emph{Natural Hazards and
Earth System Sciences} \textbf{21}, 2379--2405.

\leavevmode\vadjust pre{\hypertarget{ref-Chelton1982}{}}%
Chelton, D.B., Bernal, P.A. and McGowan, J.A. (1982) Large-scale
interannual physical and biological interaction in the california
current. \emph{Journal of Marine Research} \textbf{40}, 1095--1125.

\leavevmode\vadjust pre{\hypertarget{ref-Chen2024}{}}%
Chen, Z., Lu, Y., Zhang, J. and Zhu, W. (2024)
\href{https://doi.org/10.1287/mnsc.2023.4902}{Managing weather risk with
a neural network-based index insurance}. \emph{Management Science}
\textbf{70}, 4306--4327.

\leavevmode\vadjust pre{\hypertarget{ref-Clarke2016}{}}%
Clarke, D.J. (2016) \href{https://doi.org/10.1257/mic.20140103}{A theory
of rational demand for index insurance}. \emph{Journal: Microeconomics}
\textbf{8}, 283--306.

\leavevmode\vadjust pre{\hypertarget{ref-Dalhaus2016}{}}%
Dalhaus, T. and Finger, R. (2016)
\href{https://doi.org/10.1175/WCAS-D-16-0020.1}{Can gridded
precipitation data and phenological observations reduce basis risk of
weather index-based insurance?} \emph{Weather, Climate, and Society}
\textbf{8}, 409--419.

\leavevmode\vadjust pre{\hypertarget{ref-Feng2019}{}}%
Feng, P., Wang, B., Liu, D.L., Waters, C. and Yu, Q. (2019)
\href{https://doi.org/10.1016/j.agrformet.2019.05.018}{Incorporating
machine learning with biophysical model can improve the evaluation of
climate extremes impacts on wheat yield in south-eastern australia}.
\emph{Agricultural and Forest Meteorology} \textbf{275}, 100--113.

\leavevmode\vadjust pre{\hypertarget{ref-Free2022}{}}%
Free, C.M., Poulsen, C.V., Bellquist, L.F., Wassermann, S.N. and Oken,
K.L. (2022) \href{https://doi.org/10.1016/j.ecoinf.2022.101599}{The
CALFISH database: A century of california's non-confidential fisheries
landings and participation data}. \emph{Ecological Informatics}
\textbf{69}, 101599.

\leavevmode\vadjust pre{\hypertarget{ref-Goodrich2019}{}}%
Goodrich, B., Yu, J. and Vandeveer, M. (2019)
\href{https://doi.org/10.1057/s41288-019-00149-3}{Participation patterns
of the rainfall index insurance for pasture, rangeland and forage
programme}. \emph{The Geneva Papers on Risk and Insurance - Issues and
Practice} \textbf{45}, 29--51.

\leavevmode\vadjust pre{\hypertarget{ref-Holland2020}{}}%
Holland, D.S. and Leonard, J. (2020)
\href{https://doi.org/10.1016/j.hal.2020.101904}{Is a delay a disaster?
Economic impacts of the delay of the california dungeness crab fishery
due to a harmful algal bloom}. \emph{Harmful Algae} \textbf{98}.

\leavevmode\vadjust pre{\hypertarget{ref-Huyer1983}{}}%
Huyer, A. (1983)
\href{https://doi.org/10.1016/0079-6611(83)90010-1}{Coastal upwelling in
the california current system}. \emph{Progress in Oceanography}
\textbf{12}, 259--284.

\leavevmode\vadjust pre{\hypertarget{ref-Jacox2018}{}}%
Jacox, M.G., Edwards, C.A., Hazen, E.L. and Bograd, S.J. (2018)
\href{https://doi.org/10.1029/2018JC014187}{Coastal upwelling revisited:
Ekman, bakun, and improved upwelling indices for the u.s. West coast}.
\emph{Journal of Geophysical Research: Oceans} \textbf{123}, 7332--7350.

\leavevmode\vadjust pre{\hypertarget{ref-Jardine2020}{}}%
Jardine, S.L., Fisher, M.C., Moore, S.K. and Samhouri, J.F. (2020)
\href{https://doi.org/10.1016/j.ecolecon.2020.106691}{Inequality in the
economic impacts from climate shocks in fisheries: The case of harmful
algal blooms}. \emph{Ecological Economics} \textbf{176}.

\leavevmode\vadjust pre{\hypertarget{ref-Jensen2016}{}}%
Jensen, N.D., Barrett, C.B. and Mude, A.G. (2016)
\href{https://doi.org/10.1093/ajae/aaw046}{Index insurance quality and
basis risk: Evidence from northern kenya}. \emph{American Journal of
Agricultural Economics} \textbf{98}, 1450--1469.

\leavevmode\vadjust pre{\hypertarget{ref-Jensen2019}{}}%
Jensen, N., Stoeffler, Q., Fava, F., et al. (2019)
\href{https://doi.org/10.1016/J.ECOLECON.2019.04.014}{Does the design
matter? Comparing satellite-based indices for insuring pastoralists
against drought}. \emph{Ecological Economics} \textbf{162}, 59--73.

\leavevmode\vadjust pre{\hypertarget{ref-Keller2022}{}}%
Keller, J.B. and Saitone, T.L. (2022)
\href{https://doi.org/10.1111/ajae.12282}{Basis risk in the pasture,
rangeland, and forage insurance program: Evidence from california}.
\emph{Amer. J. Agr. Econ} \textbf{104}, 1203--1223.

\leavevmode\vadjust pre{\hypertarget{ref-Kenduiywo2021}{}}%
Kenduiywo, B.K., Carter, M.R., Ghosh, A. and Hijmans, R.J. (2021)
\href{https://doi.org/10.1371/journal.pone.0258215}{Evaluating the
quality of remote sensing products for agricultural index insurance}.
\emph{PLoS ONE} \textbf{16}.

\leavevmode\vadjust pre{\hypertarget{ref-Lehodey2006}{}}%
Lehodey, P., Alheit, J., Barange, M., et al. (2006) Climate variability,
fish, and fisheries.

\leavevmode\vadjust pre{\hypertarget{ref-Leppert2021}{}}%
Leppert, D., Dalhaus, T. and Lagerkvist, C.J. (2021)
\href{https://doi.org/10.1175/WCAS-D-20-0070.1}{Accounting for
geographic basis risk in heat index insurance: How spatial interpolation
can reduce the cost of risk}. \emph{Weather, Climate, and Society}
\textbf{13}, 273--286.

\leavevmode\vadjust pre{\hypertarget{ref-Lichtenberg2022}{}}%
Lichtenberg, E. and Iglesias, E. (2022)
\href{https://doi.org/10.1016/j.jdeveco.2022.102883}{Index insurance and
basis risk: A reconsideration}. \emph{Journal of Development Economics}
\textbf{158}.

\leavevmode\vadjust pre{\hypertarget{ref-Mumford2009}{}}%
Mumford, J.D., Leach, A.W., Levontin, P. and Kell, L.T. (2009)
\href{https://doi.org/10.1093/icesjms/fsp100}{Insurance mechanisms to
mediate economic risks in marine fisheries}. \emph{ICES Journal of
Marine Science} \textbf{66}, 950--959.

\leavevmode\vadjust pre{\hypertarget{ref-Ovando2022}{}}%
Ovando, D., Cunningham, C., Kuriyama, P., Boatright, C. and Hilborn, R.
(2022) \href{https://doi.org/10.1139/cjfas-2021-0287}{Improving
forecasts of sockeye salmon (oncorhynchus nerka) with parametric and
nonparametric models}. \emph{Canadian Journal of Fisheries and Aquatic
Sciences} \textbf{79}, 1198--1210.

\leavevmode\vadjust pre{\hypertarget{ref-Privitera2020}{}}%
Privitera-Johnson, K.M. and Punt, A.E. (2020)
\href{https://doi.org/10.1016/j.fishres.2020.105503}{A review of
approaches to quantifying uncertainty in fisheries stock assessments}.
\emph{Fisheries Research} \textbf{226}, 105503.

\leavevmode\vadjust pre{\hypertarget{ref-Rahman2022}{}}%
Rahman, L.F., Marufuzzaman, M., Alam, L., Bari, M.A., Sumaila, U.R. and
Sidek, L.M. (2022)
\href{https://doi.org/10.1007/s40009-022-01110-0}{Application of machine
learning to investigate the impact of climatic variables on marine fish
landings}. \emph{National Academy Science Letters} \textbf{45},
245--248.

\leavevmode\vadjust pre{\hypertarget{ref-Schmidt2022}{}}%
Schmidt, L., Odening, M., Schlanstein, J. and Ritter, M. (2022)
\href{https://doi.org/10.1016/J.AGSY.2021.103345}{Exploring the
weather-yield nexus with artificial neural networks}. \emph{Agricultural
Systems} \textbf{196}.

\leavevmode\vadjust pre{\hypertarget{ref-Schroeder2022}{}}%
Schroeder, I.D., Santora, J.A., Mantua, N., et al. (2022)
\href{https://doi.org/10.1016/j.ecolind.2022.109520}{Habitat compression
indices for monitoring ocean conditions and ecosystem impacts within
coastal upwelling systems}. \emph{Ecological Indicators} \textbf{144},
109520.

\leavevmode\vadjust pre{\hypertarget{ref-Sethi2010}{}}%
Sethi, S.A. (2010)
\href{https://doi.org/10.1111/j.1467-2979.2010.00363.x}{Risk management
for fisheries}. \emph{Fish and Fisheries} \textbf{11}, 341--365.

\leavevmode\vadjust pre{\hypertarget{ref-Stigler2024}{}}%
Stigler, M. and Lobell, D. (2024)
\href{https://doi.org/10.1111/ajae.12375}{Optimal index insurance and
basis risk decomposition: An application to kenya}. \emph{American
Journal of Agricultural Economics} \textbf{106}, 306--329.

\leavevmode\vadjust pre{\hypertarget{ref-Watson2023}{}}%
Watson, J.R., Spillman, C.M., Little, L.R., Hobday, A.J. and Levin, P.S.
(2023) \href{https://doi.org/10.1093/icesjms/fsad175}{Enhancing the
resilience of blue foods to climate shocks using insurance}. \emph{ICES
Journal of Marine Science} \textbf{80}, 2457--2469.

\leavevmode\vadjust pre{\hypertarget{ref-Yu2019}{}}%
Yu, J., Vandeveer, M., Volesky, J.D. and Harmoney, K. (2019)
\href{https://doi.org/10.22004/ag.econ.281319}{Estimating the basis risk
of rainfall index insurance for pasture, rangeland, and forage}.
\emph{Journal of Agricultural and Resource Economics} \textbf{44},
179--193.

\end{CSLReferences}



\end{document}
